\subsection{Efeitos coletivos}
Quando há um número suficientemente grande de elétrons em um \textit{bunch}, as interações entre os elétrons são relevantes (seja entre os elétrons ou entre os \textit{bunches}). 
	
\begin{itemize}
	\item \textit{Touschek-effect}. Dois elétrons oscilando em um \textit{bunch} podem transferir um pouco da sua energia de oscilação de uma coordenada transversal para uma longitudinal se sofrerem espalhamento de Coulomb. As novas amplitudes podem estar fora da aceitância em energia, ou aumentar o tamanho do \textit{bunch}. Esse efeito é relevante em baixas energias (menor que 1 GeV).
    \item Oscilações coerentes. Cada elétron no feixe produz campos eletromagnéticos na câmara de vácuo 	que influenciam o movimento dos outros elétrons. Estas interações coletivas podem gerar oscilações 		coerentes instáveis, em que todos os elétrons de um \textit{bunch} oscilam num modo coletivo em que 	a amplitude aumenta exponencialmente com o tempo. Estas oscilações coerentes envolvem tanto a 			dinâmica transversal quanto a longitudinal, podendo aumentar o tamanho do \textit{bunch} ou levar à 	perda de elétrons.
\end{itemize}
	
Interferência construtiva dos campos de radiação dos elétrons em um \textit{bunch} talvez gere radiação síncrona coerente, que pode aumentar a perda de energia de cada elétron individualmente. Este efeito não é considerado significante nos anéis de armazenamento mais novos.
	
Para conseguir a alta densidade de corrente desejada nos anéis de armazenamento, as instabilidades coerentes devem ser suprimidas ou controladas. Os outros efeitos coletivos são combinados com os efeitos individuais para determinar a dimensão do \textit{bunch}.