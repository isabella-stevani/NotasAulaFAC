\subsection{Separação do movimento radial}\label{sec:2.4}
É conveniente separar o movimento radial em duas partes: uma parte sendo uma curva fechada deslocada da órbita de design -- a órbita de equilíbrio dos elétrons com desvio de energia -- e a outra parte sendo a oscilação transversal em torno desta órbita. Suponha que $x$ seja
	
\begin{align}
	x = x_{\epsilon} + x_{\beta}\label{eq:2.25}
\end{align}
então certamente a equação \eqref{eq:2.19} é satisfeita se as equações
\begin{align}
	x_\epsilon'' &= -K_x(s)x_\epsilon + G(s)\frac{\epsilon}{E_0}\label{eq:2.26}\\
	x_\beta'' &= -K_x(s)x_\beta\label{eq:2.27}
\end{align}
forem verdadeiras.
	
\begin{proof}
    Pela equação \eqref{eq:2.25}, $x = x_{\epsilon} + x_{\beta}$. Logo,
    \begin{align*}
        x'' &= x_{\epsilon}'' + x_{\beta}''\\
            &= -K_x(s)x_\epsilon + G(s)\frac{\epsilon}{E_0} + K_x(s)x_\beta\\
            &= -K_x(s)(x_\epsilon + x_\beta) + G(s)\frac{\epsilon}{E_0}\\
            &= -K_x(s)x + G(s)\frac{\epsilon}{E_0}
    \end{align*}
\end{proof}
	
Definindo que $x_\epsilon(s)$ é uma função periódica em $s$ com período $L$, então $x_\epsilon(s)$ é a órbita fechada de um elétron com energia $E_0 + \epsilon$ (com $x_\beta=0$), e o movimento radial será a soma do desvio dessa nova órbita de equilíbrio e uma oscilação betatron.
	
O desvio $x_\epsilon$ é proporcional ao desvio de energia $\epsilon$. Define-se	
\begin{align}
	x_\epsilon(s) = \eta(s)\frac{\epsilon}{E_0}\label{eq:2.28}
\end{align}
onde $\eta(s)$ é a função única que satisfaz	
\begin{align}
	\begin{cases}
		\eta'' = -K_x(s)\eta + G(s), \\
        \eta(0) = \eta(L), \\
        \eta'(0) = \eta'(L).
    \end{cases}
\end{align}
	
\begin{proof}
	Seja $x_\epsilon(s)$ dado pela equação \eqref{eq:2.28}. Pela equação \eqref{eq:2.26},
	\begin{align*}
        x_\epsilon'' &= -K_x(s)x_\epsilon + G(s)\frac{\epsilon}{E_0}\\
        \left(\eta(s)\frac{\epsilon}{E_0}\right)'' &= -K_x(s)\eta(s)\frac{\epsilon}{E_0} + G(s)\frac{\epsilon}{E_0}\\
        \eta(s)''\frac{\epsilon}{E_0} &= -K_x(s)\eta(s)\frac{\epsilon}{E_0} + G(s)\frac{\epsilon}{E_0}\\
        \eta'' &= -K_x(s)\eta + G(s)
	\end{align*}
\end{proof}
	
Denomina-se $\eta(s)$ como sendo a função de órbita fechada, e é uma função que caracteriza o campo guia total do anel. Note que $\eta (s)$ é a solução particular da equação diferencial \eqref{eq:2.26} em que a função é periódica com período $L$ e não depende de características do elétron, apenas da ótica do anel.