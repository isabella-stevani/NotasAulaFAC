\subsection{Perda de energia}\label{sec:4.1}
Um elétron relativístico acelerado em um campo macroscópico irá radiar energia eletromagnética numa taxa proporcional ao quadrado da força de aceleração. Esta taxa depende do ângulo entre a força e a velocidade do elétron e é maior por um fator $\gamma^2 = (E/mc^2)^2$ quando a força é perpendicular à velocidade em comparação a quando ela é paralela à velocidade. Em um anel de armazenamento, as forças longitudinais típicas (advindas do sistema de aceleração) são muito menores que as forças transversais típicas e $\gamma^2$ é um número grande de fato, então apenas é necessário considerar os efeitos de radiação correspondentes às forças magnéticas.

Seja $P_\gamma$ a taxa de perda de energia por radiação, então pode-se escrever que
\begin{align}
	p_\gamma = \frac{2}{3}\frac{r_e c}{(mc^2)^3} E^2 F_\perp^2
\end{align}
onde $m$ é a massa de repouso do elétron, $r_e$ é o raio clássico do elétron e $F_\perp$ é a força magnética sobre o elétron. Será conveniente definir a constante
\begin{align}
	C_\gamma = \frac{4\pi}{3}\frac{r_e}{(mc^2)^3} = 8.85 \times 10^{-5}\ metro-GeV^{-3}
\end{align}
Pela força de Lorentz, $F_\perp = ecB$. Desta forma, a energia radiada é
\begin{align}
	P_\gamma = \frac{e^2c^3}{2\pi}C_\gamma E^2 B^2\label{eq:4.3}
\end{align}
Esta potência instantânea é proporcional ao quadrado tanto da energia quanto da força do campo magnético local. Às vezes é útil expressar  a força magnética em termos do raio de curvatura local $\rho$ da trajetória; então
\begin{align}
	P_\gamma = \frac{c\ C_\gamma}{2\pi}\frac{E^4}{\rho^2}
\end{align}

Um elétron circulando na órbita ideal tem energia nominal $E_0$ e se move com o raio de curvatura $\rho_s = 1/G$ -- veja a \autoref{sec:2.2}. Para encontrar a energia $U_0$ radiada em uma revolução, basta integrar $P_\gamma$ com relação ao tempo  ao longo do anel. Como $dt = ds/c$,
\begin{align}
	U_0 = \frac{C_\gamma E_0^4}{2\pi}\int\limits_{0}^{L}G^2(s)ds\label{eq:4.5}
\end{align}

Pode-se aproximar a integral pela média de $G^2$ multiplicada por $L=2\pi R$, o comprimento do anel:
\begin{align}
	U_0 = C_\gamma E_0^4 R \mean{G^2}
\end{align}
Para um campo guia isomagnético\footnote{Veja a \autoref{sec:2.2}.} $G = G_0=1/\rho_0$ em partes curvas de tamanho $2\pi\rho_0$ e zero nas demais partes, então
\begin{align}
	\mean{G^2} = \frac{G_0}{R} = \frac{1}{R \rho_0}\ \ (isomag)
\end{align}
e
\begin{align}
	U_0 = \frac{C_\gamma E_0^4}{\rho_0}\ \ (isomag)
\end{align}
\begin{proof}
	A aproximação feita anteriormente foi
	\begin{align*}
		\int\limits_{0}^{L}G^2(s)ds = \mean{G^2}L \therefore \mean{G^2} = \frac{1}{L}\int\limits_{0}^{L}G^2(s)ds
	\end{align*}
	Agora, para o caso isomagnético, $G=G_0=1/\rho_0$ nos ímãs de comprimento $2\pi \rho_0$ e zero no resto do anel. Logo, a integral fica
	\begin{align*}
		\int\limits_{0}^{L}G^2(s)ds = \int\limits_{0}^{2\pi\rho_0}G_0^2(s)ds &= 2\pi\rho_0\ G_0^2 = 2\pi\rho_0 \frac{1}{\rho_0^2} = 2\pi \frac{1}{\rho_0}\\
		\therefore \mean{G^2} &= \frac{2\pi}{L} \frac{1}{\rho_0} = \frac{1}{R \rho_0}
	\end{align*}
	Substituindo em $U_0$,
	\begin{align*}
		U_0 = C_\gamma E_0^4 R \mean{G^2} = C_\gamma E_0^4 R \frac{1}{R \rho_0} = \frac{C_\gamma E_0^4}{\rho_0}
	\end{align*}
	c.q.d.
\end{proof}

Para um raio fixo $\rho_0$, a energia radiada por revolução varia com a quarta potência da energia do elétron. A potência média radiada é $U_0/T_0$ onde $T_0 = c/2\pi R$ é o tempo de uma revolução. Para um campo guia qualquer
\begin{align}
	\mean{P_\gamma} = \frac{c C_\gamma}{2\pi} E_0^4 \mean{G^2}
\end{align} 
E para um anel isomagnético,
\begin{align}
	\mean{P_\gamma} = \frac{c C_\gamma}{2\pi}\frac{E_0^4 G_0}{R} = \frac{c C_\gamma E_0^4}{L \rho_0}\ \ (isomag)
\end{align}

Um elétron que não está na órbita ideal perde energia a uma taxa diferente. Considere primeiro um elétron com energia nominal circulando com oscilações betatron. Sua taxa de radiação será diferente da de um elétron movendo-se na órbita ideal somente porque ele passa por um campo magnético ligeiramente diferente -- devido ao seu deslocamento betatron. Mas em cada azimutal seu desvio é igualmente positivo ou negativo. E foi assumido que os campos variam apenas de forma linear com o deslocamento. Então, analisando a amplitude betatron considerando termos de até primeira ordem, a potência radiada média em um ciclo betatron é a mesma de um elétron na órbita ideal.

O mesmo não é verdadeiro para um elétron com energia diferente de $E_0$. Este caso será analisado a seguir.

Para elétrons ultra-relativísticos, a radiação é emitida em primitivamente na direção do movimento. A maioria da radiação é emitida com um ângulo $1/\gamma$. A força de reação da radiação -- e, portanto, a mudança de momento associada -- é exatamente na direção oposta do movimento\footnote{Desprezando efeitos quânticos. Veja a \autoref{sec:5.1}.}. O único efeito da radiação é diminuir a energia do elétron, sem mudar a direção do seu movimento.