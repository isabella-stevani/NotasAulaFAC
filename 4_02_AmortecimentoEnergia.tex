\subsection{Amortecimento por oscilações de energia}
Na \autoref{sec:3.5} foi visto que pequenas oscilações de energia eram amortecidas a uma taxa proporcional à mudança na perda por radiação com a energia. Pelas equações \eqref{eq:3.24} e \eqref{eq:3.42}, o coeficiente de amortecimento é
\begin{align}
	\alpha_\epsilon = \frac{D}{2T_0} = \frac{1}{2T_0}\left(\frac{dU_{rad}}{dE}\right)_0
\end{align}
onde $U_{rad}$ é a energia perdida por revolução. Quando a energia de um elétron varia da energia nominal $E_0$, a energia radiada em uma revolução muda em parte por causa da mudança de energia, em parte porque o elétron viaja em um campo magnético diferente, e em parte porque o tamanho da sua trajetória é diferente. Hora de analisar como $U_{rad}/dE$ deve ser avaliada.

Já foi visto que a oscilação betatron não muda, em uma análise de primeira ordem, a potência média radiada. Então, para obter $U_{rad}$ para qualquer energia, basta apenas integrar $P_\gamma$ da equação \eqref{eq:4.3} com relação ao tempo em uma órbita fechada completa. No entanto, será conveniente mudar a variável de integração para $s$. Assim,
\begin{align}
	U_{rad} = \oint P_\gamma dt = \oint P_\gamma \frac{dt}{ds}ds
\end{align}
lembrando que esta análise está sobre o comprimento da órbita fechada, o qual não é necessariamente o comprimento da órbita ideal. Pois bem, $dt/ds$ já foi avaliado anteriormente, veja equação  \eqref{eq:2.15}:
\begin{align*}
	\frac{dt}{ds} = \frac{1}{c} \left(1+\frac{x}{\rho_s}\right)
\end{align*}
onde $x$ é o deslocamento com relação à órbita ideal e $\rho_s = \rho(s)$ o raio de curvatura da órbita ideal. Como o interesse está sobre a perda de energia em uma órbita fechada, deve-se tomar $x = \eta \epsilon/E_0$, onde $\epsilon = E-E_0$ e $\eta(s)$ é a função de dispersão. Veja a equação \eqref{eq:2.28}. Logo,
\begin{align}
	U_{rad} = \frac{1}{c} \oint \left(1+ \frac{\eta}{\rho}\frac{\epsilon}{E_0}\right)P_\gamma ds
\end{align}

Esta integral já foi analisada para $\epsilon=0$; é apenas $U_0$. Então pode-se fazer diferente a analisar a derivada em $\epsilon=0$:
\begin{align}
	\frac{dU_{rad}}{dE} = \frac{1}{c} \oint \left(\frac{dP_\gamma}{dE}+ \frac{\eta}{\rho}\frac{P_\gamma}{E_0}\right)_0 ds\label{eq:4.14}
\end{align}
onde o subscrito "0" significa que todas as quantidades do integrando est]ao sendo avaliadas na órbita ideal, e na energia $E_0$. Pela equação \eqref{eq:4.3} $P_\gamma$ é porporcional ao produto $E^2B^2$ -- e lembre-se que quando $E$ muda, a órbita se move para uma posição diferente então $B$ também muda. Então pode-se escrever que
\begin{align*}
	\frac{dP_\gamma}{dE} = 2\frac{P_\gamma}{E_0} + 2\frac{P_\gamma}{B_0}\frac{dB}{dE}
\end{align*}

\begin{proof}
	$P_\gamma$ é dado por
	\begin{align*}
		P_\gamma = \frac{e^2c^3}{2\pi}C_\gamma E^2 B^2
	\end{align*}
	Logo, $dP_\gamma/dE$ é
	\begin{align*}
		\frac{dP_\gamma}{dE} = \frac{d}{dE}\left(\frac{e^2c^3}{2\pi}C_\gamma E^2 B^2\right) = \frac{e^2c^3}{2\pi}C_\gamma \frac{d}{dE}(E^2B^2)
	\end{align*}
	Derivando $B^2E^2$ com relação à energia, tem-se que
	\begin{align*}
		\frac{d}{dE}(E^2B^2) = 2EB^2\frac{dE}{dE} + 2E^2B\frac{dB}{dE} = 2EB^2 + 2E^2B\frac{dB}{dE}
	\end{align*}
	Substituindo, na expressão de $dP_\gamma/dE$:
	\begin{align*}
		\frac{dP_\gamma}{dE} = \frac{e^2c^3}{2\pi}C_\gamma \left(2EB^2 + 2E^2B\frac{dB}{dE}\right) = 2\frac{e^2c^3}{2\pi}C_\gamma E B^2 + 2\frac{e^2c^3}{2\pi}C_\gamma E^2 B \frac{dB}{dE}
	\end{align*}
	Pode-se substituir a equação de $P_\gamma$, obtendo-se
	\begin{align*}
		\frac{dP_\gamma}{dE} = 2\frac{P_\gamma}{E} + 2\frac{P_\gamma}{B}\frac{dB}{dE}
	\end{align*}
	Avaliando a expressão na órbita ideal,
	\begin{align*}
		\frac{dP_\gamma}{dE} = 2\frac{P_\gamma}{E_0} + 2\frac{P_\gamma}{B_0}\frac{dB}{dE}
	\end{align*}
\end{proof}

Mas
\begin{align*}
	\frac{dB}{dE} = \frac{dx}{dE}\frac{dB}{dx} = \frac{\eta}{E_0}\frac{dB}{dx}
\end{align*}
onde $dB/dx$ é uma propriedade do campo guia. Juntando as duas últimas equações na equação \eqref{eq:4.14}, tem-se que
\begin{align*}
	\frac{dU_{rad}}{dE} = \frac{1}{c}\oint \left(2\frac{P_\gamma}{E} + 2\frac{P_\gamma}{B}\frac{\eta}{E_0}\frac{dB}{dx} + \frac{P_\gamma}{E}\frac{\eta}{\rho}\right)_0 ds
\end{align*}

A integral do primeiro termo é apenas $2U_0/E_0$, então
\begin{align}
	\frac{dU_{rad}}{dE} = \frac{U_0}{E_0}\left[2+\frac{1}{cU_0}\oint \left(\eta P_\gamma \left[\frac{1}{\rho} + \frac{2}{B}\frac{dB}{dx}\right]\right)_0 ds\right]
\end{align}

Então pode-se escrever que a constante de amortecimento é dada por
\begin{align}
	\alpha_\epsilon = \frac{1}{2T_0}\left(\frac{dU_{rad}}{dE}\right)_0 = \frac{U_0}{2T_0E_0}(2+\mathscr{D})\label{eq:4.16}
\end{align}
com
\begin{align}
	\mathscr{D} = \frac{1}{cU_0}\oint \left[\eta P_\gamma\left(\frac{1}{\rho}+\frac{2}{B}\frac{dB}{dx}\right)\right]_0 ds\label{eq:4.17}
\end{align}
Tomando $P_\gamma$ e $U_0$ das equações \eqref{eq:4.3} e \eqref{eq:4.5} e expressando $B$ e $dB/dx$ em termos de $G(s)$ e $K_1(s)$ assim como foram definidos na \autoref{sec:2.2}, $\mathscr{D}$ pode ser reescrito como
\begin{align}
	\mathscr{D} = \frac{\oint \eta G \left(G^2 + 2K_1\right)ds}{\oint G^2 ds}\label{eq:4.18}
\end{align}

\begin{proof}
	Pelas equações \eqref{eq:4.3} e \eqref{eq:4.5}, tem-se que
	\begin{align*}
		P_\gamma = \frac{e^2c^3}{2\pi}C_\gamma E^2B^2\\
		U_0 = \frac{C_\gamma E_0^4}{2\pi}\oint G^2 ds
	\end{align*}
	Substituindo na equação \eqref{eq:4.17}:
	\begin{align*}
		\mathscr{D} &= \frac{2\pi}{c C_\gamma E_0^4}\frac{1}{\oint G^2 ds}\oint \left[\eta \frac{e^2c^3}{2\pi}C_\gamma E^2B^2 \left(\frac{1}{\rho}+\frac{2}{B}\frac{dB}{dx}\right)\right]_0 ds\\
					&= \frac{\oint \left[\eta \frac{e^2c^2}{E_0^2} B^2 \left(\frac{1}{\rho}+\frac{2}{B}\frac{dB}{dx}\right)\right]_0 ds}{\oint G^2 ds}
	\end{align*}
	Agora, recuperando os valores de $G(s)$ e $K_1(s)$ da \autoref{sec:2.2}, pode-se substituí-los e obter
	\begin{align*}
		\mathscr{D} &= \frac{\oint \left[\eta G^2 \left(G +\frac{2ec}{G E_0}\frac{K_1 E_0}{ec}\right)\right]_0 ds}{\oint G^2 ds}\\
					&= \frac{\oint \left[\eta G^2 \left(G +\frac{2K_1}{G}\right)\right]_0 ds}{\oint G^2 ds}\\
		\therefore \mathscr{D} &= \frac{\oint \left[\eta G \left(G^2 +2K_1\right)\right]_0 ds}{\oint G^2 ds}
	\end{align*}
	c.q.d.
\end{proof}

Essa forma torna claro o fato de que $\mathscr{D}$ é um número o qual é uma propriedade da configuração total do campo guia -- obtido da integração ao redor do anel das expressões envolvendo apenas as funções do campo guia $G$, $K_1$ e $\eta$. O número $\mathscr{D}$ tipicamente é positivo e um pouco maior que 1.

A equação \eqref{eq:4.16} tem uma interpretação física relevante. Já que $\mathscr{D}$ é tipicamente pequeno, pode-se aproximar a relação
\begin{align}
	\alpha_\epsilon \approx \frac{U_0}{E_0T_0} = \frac{\mean{P_\gamma}}{E_0}
\end{align}
onde $\mean{P_\gamma}$ é a taxa média de perda de radiação. A constante de tempo de amortecimento para oscilações de energia -- a qual é o inverso de $\alpha_\epsilon$ -- é apenas o tempo necessário para um elétron radiar toda a sua energia!

A expressão de $\mathscr{D}$ se torna mais simples para um campo guia isomagnético. Neste caso, $G(s)$ é zero ou um valor constante $G_0$ nos ímãs, então as integrais são avaliadas apenas nos ímãs. A equação \eqref{eq:4.18} torna-se
\begin{align}
	\mathscr{D} = \frac{1}{2\pi} \int\limits_{mag}^{}\eta(s)\left[G_0^2 + 2K_1(s)\right]ds\ \ (isomag)
\end{align}

\begin{proof}
	Como já foi dito, $G(s)$ é zero ou um valor constante $G_0$ nos ímãs, então as integrais são avaliadas apenas nos ímãs. Logo, $\mathscr{D}$ torna-se
	\begin{align*}
		\mathscr{D} &= \frac{\int\limits_{mag}^{} \eta G_0 \left(G_0^2 + 2K_1\right)ds}{\int\limits_{mag}^{} G_0^2 ds} = \frac{G_0\int\limits_{mag}^{} \eta \left(G_0^2 + 2K_1\right)ds}{2\pi \rho_0 G_0^2}\\
		 &= \frac{1}{2\pi \rho_0 G_0}\int\limits_{mag}^{} \eta \left(G_0^2 + 2K_1\right)ds
	\end{align*}
	Mas $G_0 = 1/\rho_0$. Então,
	\begin{align*}
		\mathscr{D} = \frac{1}{2\pi}\int\limits_{mag}^{} \eta \left(G_0^2 + 2K_1\right)ds
	\end{align*}
\end{proof}

Se o campo guia também é de função separável, os ímãs não possuem gradiente de campo e
\begin{align}
	\mathscr{D} = \frac{G_0^2}{2\pi}\int\limits_{mag}^{}\eta(s)ds\ \ (isomag.\ e\ fun. sep.)
\end{align}
esta integral é familiar: ela apareceu anteriormente quando o fator de compactação de momento $\alpha$ foi calculado para um campo guia isomagnético. Utilizando as equações \eqref{eq:3.13} e \eqref{eq:3.14}:
\begin{align}
	\mathscr{D} = G_0\mean{\eta}_{mag} = G_0\alpha R = \frac{\alpha R}{\rho_0}\ \ (isomag.\ e\ fun. sep.)
\end{align}
Para este tipo de anel, o número $\mathscr{D}$ é apenas o fator de compactação de momento aumentado por uma razão entre o raio de curvatura típico $R$ e o raio de curvatura do ímã $\rho_0$. Alguns valores típicos para estes parâmetros:
\begin{align}
	\alpha \approx 0.05;\ \ \ R/\rho_0 \approx 3;\ \ \ \mathscr{D} \approx 0.15.
\end{align}
Recapitulando, para oscilações de energia em um campo guia isomagnético e de função separável, o coeficiente de amortecimento por oscilações de energia é
\begin{align}
	\alpha_\epsilon = \frac{\mean{P_\gamma}}{2E_0}\left(2+ \frac{\alpha R}{\rho_0}\right)\ \ (isomag.\ e\ fun. sep.)
\end{align}