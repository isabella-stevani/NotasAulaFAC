\subsection{Comprimento do \textit{bunch}}
Foi visto que a distribuição do desvio de tempo normalizado $\theta$ é Gaussiana com desvio padrão igual ao desvio padrão das oscilações de energia $\sigma_\epsilon$ -- veja a equação \eqref{eq:5.57}. Segue que a flutuação das oscilações de energia é acompanhada de flutuações associadas ao desvio de tempo $\tau$, e o desvio padrão $\sigma_\tau$ dessas flutuações é -- veja a equação \eqref{eq:5.52} --
\begin{align}
	\sigma_\tau = \frac{\alpha}{\Omega E_0}\sigma_\epsilon
\end{align}
Para um campo guia isomagnético, a equação \eqref{eq:5.48} leva a
\begin{align}
	\sigma_\tau^2 = \frac{\alpha^2}{\Omega^2} \frac{C_q \gamma_0^2}{J_\epsilon \rho_0}\ \ (isomag.)
\end{align}
Tomando $\Omega^2$ da equação \eqref{eq:3.44},
\begin{align}
	\sigma_\tau^2 = \frac{\alpha T_0 E_0}{e \dot{V}_0}\frac{C_q \gamma_0^2}{J_\epsilon \rho_0} = \frac{2\pi C_q}{(mc^2)^2}\frac{\alpha R}{J_\epsilon \rho_0}\frac{E_0^3}{e \dot{V}_0}\ \ (isomag.)\label{eq:5.66}
\end{align}
O \textit{spread} $\sigma_\tau$ no desvio de tempo multiplicada por $c$ resulta no \textit{spread} do deslocamento longitudinal com relação ao centro do \textit{bunch} -- chamado também de comprimento médio do \textit{bunch}.

Se a energia $E_0$ do feixe armazenado em um anel de armazenamento em particular varia enquanto a inclinação da tensão de RF ($\dot{V}_0$) se mantém constante, o comprimento do \textit{bunch} irá aumentar com $E_0^{2/3}$. No entanto, pode ser vantajoso ajustar a tensão de RF quando a energia varia de forma que o pico da tensão de RF se mantenha proporcional a $E_0^3$ e, pela equação \eqref{eq:5.66}, o comprimento do \textit{bunch} será independente da energia. O comprimento constante do \textit{bunch} -- $2c\sigma_\tau$ neste caso -- será em torno de 10\% da distância entre os centros dos \textit{bunches}.

Em vários anéis de armazenamento que foram construídos até então, foi observado que o comprimento do \textit{bunch} tende a ser maior do que foi previsto aqui por um fator significante que depende do número de elétrons armazenado no \textit{bunch}. O mecanismo responsável por essa anomalia ainda não foi entendido.