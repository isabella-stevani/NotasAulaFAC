\documentclass[a4paper, 12pt]{./Template/lnls-note-PT}
\usepackage{amsmath}
\usepackage{amssymb}
\usepackage{amsthm}
\usepackage{indentfirst}
\numberwithin{equation}{section} %numera a equação de acordo com sua seção
\usepackage[pdftex,colorlinks=true,citecolor=black,linkcolor=black,urlcolor=black,filecolor=black]{hyperref}
\usepackage[num,abnt-emphasize=bf, bibjustif]{abntex2cite}
\usepackage{tipa}
\citebrackets[]
%\documentclass[a4paper, 12pt, report]{lnls-note} % if you want to have chapters choose this one
\newcommand{\lambdabar}{\mbox{\textipa{\textcrlambda}}}

%loads standard preamble configuration
\input{./Template/standard_preamble.tex}

%loads standard commands
\input{./Template/new_commands.tex}


\begin{document}

\lnlstitle{134518}{\today}{Grupo de Estudos: Física de Aceleradores}
{Isabella Stevani}{\LNLS}
{Seguindo o livro \textit{The physics of electron storage rings: an introdution} de Matthew Sands \cite{sands1970physics}, este grupo de estudos tem por objetivo estudar e discutir o funcionamento de um acelerador de partículas, desde os elementos que o compõe como os fenômenos físicos que modelam sua dinâmica. Este documento traz anotações e deduções úteis de forma a facilitar o completo entendimento deste assunto.}

\newpage
\setcounter{page}{2}
\tableofcontents

\newpage
\section{Introdução}
	%\input{1_ComentariosIniciais}
    \subsection{Mecanismos básicos}
Para facilitar o entendimento dos processos descritos a seguir, a Figura \ref{fig:ring} representa a estrutura física de um anel de armazenamento.
	
\begin{figure}[!htb]
	\centering
	\includegraphics[width=0.6\linewidth]{./Figuras/fig1.jpeg}
	\caption{Diagrama esquemático de um anel de armazenamento de elétrons. Retirado de \cite{sands1970physics}.}
	\label{fig:ring}
\end{figure}
	
Um feixe de elétrons é injetado em uma câmara circular em vácuo. Campos magnetostáticos guiam as partículas pela câmara de vácuo. Tais campos são chamados de \emph{campos guia}. A interação eletromagnética entre o campo guia e os elétrons causa uma aceleração centrípeta do feixe, curvando assim sua trajetória e gerando um \emph{órbita fechada} desejada e desenhada a partir da escolha criteriosa do campo.
	
Esse campo guia, além de fechar a trajetória em uma órbita, tem propriedades focalizadoras que fazem com que cada elétron execute oscilações pseudo-harmônicas na transversal, as quais são chamadas de \emph{oscilações betatron}.
	
A cada volta, o elétron perde uma pequena parte da sua energia em forma de \emph{radiação síncrotron}. Esta energia perdida é reposta através de campos elétricos oscilantes no tempo gerados na \emph{cavidade de rádio frequência} (RF) e que aceleram longitudinalmente o feixe. Essas pequenas variações de perda e ganho da energia dos elétrons causam também oscilações no comprimento de órbita, no período de revolução, e ainda, no instante de chegada $\tau$ dos elétrons na cavidade de RF. As oscilações no plano energia-$\tau$ são chamadas de \emph{oscilações síncrotron} (ou oscilações de fase). Essas oscilações longitudinais dos elétrons são medidas em relação às \emph{partículas síncronas}, que são definidas como aqueles elétrons idealizados cujas energias, períodos de revolução e fase de chegada à cavidade de RF têm valores nominais.

Existem fases ideais de chegada à cavidade de RF em que o campo elétrico da cavidade é tal que a energia reposta por ele coincide com a energia média perdida pelos elétrons em cada volta. Cada uma destas fases ideais é chamada de \emph{fase síncrona}. O \emph{número harmônico} de um anel é definido como sendo a razão inteira entre a frequência de RF e a frequência de revolução dos elétrons. O número harmônico nos dá então o número de fases síncronas que podem ser acomodados no anel. Ao redor de cada uma destas fases síncronas, que são pontos fixos do espaço de fase energia-$tau$ nos quais se situam as partícula síncronas, pode-se ter uma distribuição de elétrons descrevendo oscilações síncrotron. Estas distribuições de elétrons em torno de das fases síncronas são chamadas de pacotes ou \emph{bunches}. Tipicamente o número harmônico dos anéis assume valor alto, permitindo assim um grande número de bunches e, em consequência, uma corrente de feixe circulante mais intensa.
	
A perda de energia por radiação síncrotron junto com a compensação gerada pela cavidade de RF causa um lento amortecimento da radiação de todas as amplitudes de oscilação, fazendo com que a trajetória de cada elétron tenda à trajetória do elétron de referência, o qual possui velocidade constante ao longo da órbita ideal.
	
Amortecimento da radiação não conserva o espaço de fase, então pode-se injetar vários \textit{bunches} no mesmo anel. O amortecimento de todas as amplitudes de oscilação é efetivamente preso devido à excitação contínua das oscilações por um "ruído" na energia do elétron, que vem do fato da radiação síncrona ser emitida em fótons de energia discreta. Este fenômeno é chamado de flutuação quântica da perda de energia. Em condições estacionárias, um equilíbrio é alcançado entre a excitação quântica e o amortecimento da radiação, levando a uma distribuição estatística estacionária das amplitudes e fases de oscilação dos elétrons em um \textit{bunch}. O \textit{bunch}, então, toma forma de uma tira elástica viajante, a qual tem tamanho e forma estacionários, com uma distribuição Gaussiana das amplitudes em cada coordenada transversal e longitudinal (Figura \ref{fig:fig2}). 
	
\begin{figure}[!htb]
	\centering
	\includegraphics[width=0.55\linewidth]{./Figuras/fig2.jpeg}
	\caption{\textit{Bunches} circulando em um anel de armazenamento. Retirado de \cite{sands1970physics}.}
	\label{fig:fig2}
\end{figure}
	
Para cada coordenada do elétron, existe uma amplitude de oscilação máxima chamada de abertura dinâmica, a qual o elétron não fica mais preso dentro do \textit{bunch}. A abertura dinâmica para cada coordenada é definida por um obstáculo físico (tamanho da câmara de vácuo, por exemplo) ou por efeitos não-lineares nas forças focalizadoras, gerando trajetórias não-limitadas.
    \subsection{Efeitos coletivos}
Quando há um número suficientemente grande de elétrons em um \textit{bunch}, as interações entre os elétrons são relevantes (seja entre os elétrons ou entre os \textit{bunches}). 
	
\begin{itemize}
	\item \textit{Touschek-effect}. Dois elétrons oscilando em um \textit{bunch} podem transferir um 		pouco da sua energia de oscilação de uma coordenada para a outra se sofrerem espalhamento de 			Coulomb. As novas amplitudes podem estar fora da abertura dinâmica, ou aumentar o tamanho do 			\textit{bunch}. Esse efeito é relevante em baixas energias (menor que 1 GeV).
    \item Oscilações coerentes. Cada elétron no feixe produz campos eletromagnéticos na câmara de vácuo 	que influenciam o movimento dos outros elétrons. Estas interações coletivas podem gerar oscilações 		coerentes instáveis, em que todos os elétrons de um \textit{bunch} oscilam num modo coletivo em que 	a amplitude aumenta exponencialmente com o tempo. Estas oscilações coerentes envolvem tanto a 			dinâmica transversal quanto a longitudinal, podendo aumentar o tamanho do \textit{bunch} ou levar à 	perda de elétrons.
\end{itemize}
	
Interferência construtiva dos campos de radiação dos elétrons em um \textit{bunch} talvez gere radiação síncrona coerente, que pode aumentar a perda de energia de cada elétron individualmente. Este efeito não é considerado significante nos anéis de armazenamento mais novos.
	
Para conseguir a alta densidade de corrente desejada nos anéis de armazenamento, as instabilidades coerentes devem ser suprimidas ou controladas. Os outros efeitos coletivos são combinados com os efeitos individuais para determinar a dimensão do \textit{bunch}.
    \pagebreak
  
\section{Oscilações Betatron}
	\subsection{Sistema de coordenadas}
É conveniente utilizar um sistema de coordenadas cilíndrico para descrever a trajetória de um elétron dentro do anel de armazenamento (\autoref{fig:fig7}), uma vez que é desejado que seu movimento seja circular. Sendo assim, definem-se as coordenadas
\begin{itemize}
	\item $s$ -- Coordenada longitudinal, representa a distância entre um ponto de referência na órbita ideal e o ponto mais próximo do elétron nesta mesma órbita.
	\item $x$ e $z$ -- Coordenadas transversais, representam os deslocamentos horizontal e vertical com relação à órbita ideal, respectivamente.
\end{itemize}

\begin{figure}[!htb]
	\centering
	\includegraphics[width=0.6\linewidth]{./Figuras/fig7.jpeg}
	\caption{Coordenadas para descrever as trajetórias. Retirado de \cite{sands1970physics}.}
	\label{fig:fig7}
\end{figure}
    \subsection{Campo guia}
Pelo principio da inércia de Newton, um corpo tende a permanecer em movimento retilínio uniforme se não há forças que o obriguem a mudar sua trajetória. Logo, para que o elétron tenha uma órbita circular, é necessário aplicar uma força que mude sua trajetória da forma desejada.

Pela força de Lorentz,
\begin{align}
	\vec{F} = q\left(\vec{E} + \vec{v} \times \vec{B}\right)
\end{align}
onde $q$ é a carga da partícula em movimento, $\vec{E}$ é o vetor de campo elétrico, $\vec{v}$ é o vetor de velocidade da partícula e $\vec{B}$ é o vetor de campo magnético.

A contribuição referente à força elétrica ($q\vec{E}$) é paralela ao campo elétrico, enquanto a contribuição referente à força magnética ($q(\vec{v}\times\vec{B})$) é perpendicular ao campo magnético e à velocidade. Devido a este fato, a força magnética não realiza trabalho, uma vez que é perpendicular ao deslocamento da partícula. Logo, a força magnética altera a direção do vetor velocidade -- e, por consequência, do movimento da partícula -- sem alterar o seu módulo. Apenas a força elétrica pode realizar trabalho.

Desta forma, a fim de desviar a partícula da sua trajetória, aplica-se um campo magnético $\vec{B}$ nos pontos onde ela deve fazer alguma curva, o qual é chamado de campo guia. Com $E=0$,
\begin{align}
	\vec{F} = q\vec{v}\times\vec{B}
\end{align}

Espera-se que a órbita ideal ocorra no plano horizontal ($z=0$) de forma que o campo magnético deve ser puramente vertical em toda a órbita do anel. Considerando o campo magnético simétrico com relação ao plano da órbita ideal e fazendo uma aproximação linear, pode-se escrever a equação do campo magnético através da expansão de Taylor:
\begin{align}
	B_z(s,x,z) &= B_0(s) + \left(\frac{\partial B_z}{\partial x}\right)_{0s} x\label{eq:2.01}\\
	B_x(s,x,z) &= \left(\frac{\partial B_x}{\partial z}\right)_{0s} z\label{eq:2.02}
\end{align}

Pela simetria imposta, aplicando as leis de Maxwell,
\begin{align}
	\frac{\partial B_z}{\partial x} = \frac{\partial B_x}{\partial z}
\end{align}

Desta forma, as equações \eqref{eq:2.01} e \eqref{eq:2.02} podem ser reescritas como
\begin{align}
	B_z(s,x,z) &= B_0(s) + \left(\frac{\partial B}{\partial x}\right)_{0s} x\label{eq:2.1}\\
	B_x(s,x,z) &= \left(\frac{\partial B}{\partial x}\right)_{0s} z\label{eq:2.2}
\end{align}

Como o campo é simétrico em relação ao plano da órbita ideal, as variáveis $B_0$ e $\frac{\partial B}{\partial x}$ possuem apenas componentes verticais, então apenas suas magnitudes são necessárias para descrevê-las completamente e os índices $s$ e $z$ podem ser suprimidos.

Anéis de armazenamento são modelados para operar em uma faixa de valores de energia dos elétrons. Isto é obtido arranjando de forma que os campos magnéticos possam ser variados juntos -- sendo parametrizados proporcionalmente à energia de operação desejada. Claramente, se o campo magnético na órbita ideal é mudado em todo o anel pelo mesmo fator, a órbita ideal será, novamente, uma trajetória possível de um elétron cujo momento é mudado pelo mesmo fator. Variar todos os campos juntos altera muito pouco a energia associada com a órbita ideal. Por essas razões, é conveniente especificar as propriedades do campo guia de uma maneira que seja independente de qualquer energia de operação escolhida, o que é facilmente feito dividindo todos os campos por um fator proporcional à energia associada do elétron, o qual é chamado de rigidez magnética. Desta forma, as propriedades lineares do campo guia podem ser definidas pelas funções
\begin{align}
	G(s) &= \frac{ecB_0(s)}{E_0}\\
	K_1(s) &= \frac{ec}{E_0} \left(\frac{\partial B}{\partial x}\right)_{0s}
\end{align}
onde $E_0$ é a energia nominal, $c$ é a velocidade da luz e $e$ é a carga do elétron. Note que estas funções tem um significado físico bem simples. Neste caso, apenas elétrons ultra-relativísticos são considerados, então sua energia é dada por $E=cp$. Desta forma, $G(s)$ é apenas o inverso do raio de curvatura $\varrho(s)$ dos elétrons em $x=0$ e $z=0$ com energia nominal, ou seja,
\begin{align}
	G(s) = \frac{1}{\varrho(s)}
\end{align}

\begin{proof}
	Como a partícula está rotacionando sobre ação da força de Lorentz, esta força deve ser equivalente a sua força centrípeta. Logo,
	\begin{align*}
		\frac{mv^2}{\varrho(s)} &= q\vec{v}\times\vec{B}\\
							    &= evB_0(s)\\
	\end{align*}
	
	Como o elétron é ultra-relativístico (com velocidade próxima/igual à velocidade da luz), a energia da partícula é dada por
	\begin{align*}
		E_0^2 = (m_0 c^2)^2 + p^2c^2
	\end{align*}
	onde $m_0$ é a massa de repouso e $p$ o momento da partícula. Pelo mesmo argumento anterior, $(m_0c^2)^2 << c^2p^2$. Desta forma, pode desprezar o termo $(m_0c^2)^2$ e a energia da partícula é dada por
	
	\begin{align*}
		E_0 &= pc\\
			&= \gamma m_0 c^2\\
			&= mc^2
	\end{align*}
	
	Substituindo,
	\begin{align*}
			\frac{mv^2}{\varrho(s)} &= ecB_0(s)\\
			\frac{E_0}{\varrho(s)} &= ecB_0(s)\\
			\therefore \frac{1}{\varrho(s)} &= \frac{ecB_0(s)}{E_0} = G(s)
	\end{align*}
	
	Desta forma, reescrevendo a equação do campo magnético em função de $G(s)$,
	\begin{align*}
		B_z(s,x,z) &= B_0(s) + \frac{\partial B}{\partial x} x\\
				   &= \frac{E_0}{ec} G(s) + \frac{\partial B}{\partial x} x
	\end{align*}
	
	Definindo a função $K_1(s)$ como
	\begin{align*}
		K_1(s) = \frac{ec}{E_0} \frac{\partial B}{\partial x}
	\end{align*}
	tem-se que
	\begin{align*}
		B_z(s,x,z) &= \frac{E_0}{ec} G(s) + \frac{\partial B}{\partial x} x\\
				   &= \frac{E_0}{ec} G(s) + \frac{E_0}{ec} K_1(s) x\\
				   &= \frac{E_0}{ec} [G(s) + K_1(s) x]
	\end{align*}
	Reescrevendo $B_x(s,x,z)$ da mesma forma, as equações do campo magnético são
	\begin{align*}
		B_z(s,x,z) &= \frac{E_0}{ec} [G(s) + K_1(s) x]\\
		B_x(s,x,z) &= \frac{E_0}{ec} K_1(s) z
	\end{align*}
\end{proof}

A constante de proporcionalidade $\frac{E_0}{ec}$ é a rigidez magnética do anel de armazenamento. Ela normaliza as equações do campo magnético de forma que este dependa apenas das propriedades da ótica do anel.

Devido à relação entre $G(s)$ e $\varrho(s)$, $G(s)$ é chamada de função de curvatura. A função $K_1(s)$ é a taxa de variação do raio inverso com o deslocamento radial.

As funções $G(s)$ e $K_1(s)$ podem ser arbitrárias, porém devem satisfazer alguns requisitos importantes. Primeiro, $G(s)$ precisa ser tal que esta defina uma órbita fechada (pode-se pensar que $G(s)$ define a órbita ideal, ou que alguma órbita fechada arbitrária define $G(s)$ de forma única). A variação $d \theta_0$ na direção da tangente à órbita ideal em um intervalo $ds$ é
\begin{align}
	-d\theta_0 = \frac{ds}{\varrho(s)} = G(s)ds
\end{align}

O ângulo percorrido em uma revolução precisa ser igual a $2\pi$, então $G(s)$ deve satisfazer
\begin{align}
	\int_{0}^{L} G(s)ds = 2\pi
\end{align}

Segundo, tanto $G(s)$ quanto $K_1(s)$ são, necessariamente, funções periódicas de $s$, devido ao fato de que a coordenada longitudinal $s$ é fisicamente cíclica -- retornando ao mesmo ponto da órbita após uma revolução. Dito isso, $G(s)$ e $K_1(s)$ também devem satisfazer
\begin{align}
	\left\{\begin{array}{rcl}
	\ G(s+L) & = & G(s)\\
	K_1(s+L) & = & K_1(s)
	\end{array}\right.
\end{align}
onde $L$ é o comprimento da órbita. Execeto por estas condições, $G(s)$ e $K_1(s)$ podem ter mais ou menos variações arbitrárias com $s$.

Apesar das funções do campo guia $G$ e $K_1$ serem, a princípio, bem gerais, geralmente é conveniente simplificar o design ou a operação de um anel de armazenamento impondo certas restrições nestes aspectos. Por exemplo, a maioria dos anéis de armazenamento são desenhados para ter o mesmo raio de curvatura, diga-se $\varrho_0$, em todos os ímãs de curvatura -- e sem nenhuma curvatura entre um ímã e outro, ou seja, apenas trechos retos. Este tipo de campo guia é chamado isomagnético. O intuito desta configuração é que o campo magnético sobre a órbita ideal tenha o mesmo valor em todo lugar, exceto onde este é nulo. Então $G(s)$ é uma função dicotômica:
\begin{align}
	G(s) = \left\{\begin{array}{rrrr}
	\ G_0 & = & \frac{1}{\varrho_0}, &  no\ \acute{i}m\tilde{a}\\
	0, & & & em\ todo\ o\ resto
	\end{array}\right.
\end{align}

Um campo guia real não pode, claro, ser idealmente isomagnético, já que é fisicamente impossível ter um campo magnético descontínuo. Sempre há uma zona de transição nas bordas do ímã onde o campo vai de zero ao seu valor nominal. A aproximação isomagnética ideal é, entretanto, um tanto quanto adequada no geral.

Apesar de aceleradores e anéis de armazenamento comumente serem construídos com ímãs de curvatura com gradientes radiais de campo, é comum desenvolver campos guia de função separável, ou seja, campos magnéticos em que as funções de focalização e curvatura são atribuídas a elementos magnéticos diferentes. Isto é, o campo guia consiste numa sequência de dipolos (sem gradiente de campo) e quadrupolos (sem campo na órbita ideal). Pensando nesta configuração, define-se um campo guia de função separável onde as funções $G(s)$ e $K_1(s)$ devem satisfazer a condição
\begin{align}
	G(s)K_1(s) = 0\label{eq:2.10}
\end{align}

Um pouco de atenção para este fato. Às vezes é conveniente projetar os ímãs de curvatura com faces retangulares. Com este tipo de ímã, a órbita ideal deve entrar ou sair do ímã com um ângulo diferente de 90$^\circ$ em relação à borda do mesmo (Figura \ref{fig:fig8}). Mesmo que o ímã seja plano (sem gradiente radial no ímã), irão existir gradientes radiais nas bordas, onde o campo não é nulo. A equação \eqref{eq:2.10} não é satisfeita nas bordas, e um campo guia construído com estes ímãs retangulares -- junto com os quadrupolos -- não irá satisfazer a definição de função separável, mesmo que ele seja referenciado como tal às vezes. Estes campos ainda podem ser, entretanto, isomagnéticos. 

\begin{figure}[!htb]
	\centering
	\includegraphics[width=0.6\linewidth]{./Figuras/fig8.jpeg}
	\caption{Campo guia com ímã retangular. Retirado de \cite{sands1970physics}.}
	\label{fig:fig8}
\end{figure}
    \subsection{Equações de movimento}\label{sec:2.3}
As equações de movimento descrevem a trajetória de um elétron se movendo próximo da órbita ideal, com uma energia próxima, mas não necessariamente igual, à energia nominal $E_0$. O desvio de energia é definido como
\begin{align}
	\epsilon = E-E_0
\end{align}
onde $E$ é a energia do elétron.

Para manter a aproximação linear, são considerados apenas pequenas quantidades de $x$, $z$ e $\epsilon$. Melhor do que tomar o tempo como variável independente, é mais conveniente adotar a coordenada longitudinal $s$. Assim, derivadas com relação a $s$ serão indicadas daqui pra frente pela notação $(')$. Por exemplo, $x'=\frac{dx}{ds}$.

Começando a análise pelo movimento radial. Considere um elétron em $x$ movendo-se com inclinação $x'$ (Figura \ref{fig:fig9}). A inclinação é o ângulo entre a direção do movimento do elétron e a tangente à órbita ideal. $x'$ é o ângulo entre a trajetória e a tangente à órbita ideal. Suponha um ângulo $\theta_0$ entre a tangente e uma direção de referência arbitrária e um ângulo $\theta$ entre a trajetória e a mesma direção de referência. Logo, $x' = \theta-\theta_0$ e
\begin{align}
	x'' = \frac{d(\theta-\theta_0)}{ds}\label{eq:2.12}
\end{align}

\begin{figure}[!htb]
	\centering
	\includegraphics[width=0.6\linewidth]{./Figuras/fig9.jpeg}
	\caption{Trajetória de um elétron próxima à órbita ideal. Retirado de \cite{sands1970physics}.}
	\label{fig:fig9}
\end{figure}

A derivada de $\theta_0$ é, como já foi visto, $\frac{-1}{\varrho_s} = G(s)$ ($\varrho_s = \varrho(s)$). Mas o que é $\frac{d\theta}{ds}$? O rario de curvatura da trajetória é
\begin{align}
	\varrho = \frac{E}{ecB}
\end{align}
e, em um elemento de caminho $d\ell$ da trajetória, a variação do ângulo é
\begin{align}
	d\theta = -\frac{d\ell}{\varrho} = -\frac{ecB}{E}d\ell\label{eq:2.14}
\end{align}

Note que, enquanto o ângulo $x'$ é pequeno -- e pode-se sempre assumir isto desde que apenas termos de primeira ordem sejam considerados, que é o caso -- um elemento de caminho $d\ell$ da trajetória em $x$ é relacionado com a correspondente variação em $s$ por
\begin{align}
	d\ell = \frac{\varrho_s - x}{\varrho_s} ds = \left(1+\frac{x}{\varrho_s}\right)ds = (1+G(s))ds\label{eq:2.15}
\end{align}

Agora, $B$ pode ser descrito por
\begin{align}
	B = B_0 + \frac{\partial B}{\partial x}x = \frac{E_0}{ec}(G+K_1x)\label{eq:2.16}
\end{align}

Substituindo as equações \eqref{eq:2.15} e \eqref{eq:2.16} na equação \eqref{eq:2.14} juntamente com $E_0+\epsilon$ para $E$ -- e mantendo apenas termos de primeira ordem -- tem-se que
\begin{align}
	d\theta = \left\{-G-(G^2+K_1)x + G\frac{\epsilon}{E_0}\right\}ds
\end{align}
e, portanto, 
\begin{align}
	x'' = -(G^ 2+K_1)x + G\frac{\epsilon}{E_0}
\end{align}

\begin{proof}
	Pela equação \eqref{eq:2.14},
	\begin{align*}
		d\theta &= -\frac{ecB}{E}d\ell\\
				&= -\frac{ecB}{E}(1+Gx)ds\\
				&= -\frac{ec}{E}\left[\frac{E_0}{ec}(G+K_1x)\right](1+Gx)ds\\
				&= -\frac{E_0}{E_0+\epsilon}(G+K_1x)(1+Gx)ds\\
				&= -\frac{E_0}{E_0+\epsilon}(G+K_1x+G^2x+K_1Gx^2)ds\\
				&= -\frac{E_0}{E_0+\epsilon}(G+(G^2+K_1)x + K_1Gx^2)ds
	\end{align*}
	
	Mantendo apenas termos de primeira ordem para continuar com a aproximação linear,
	\begin{align*}
		d\theta &= -\frac{E_0}{E_0+\epsilon}(G+(G^2+K_1)x)ds\\
				&= \frac{E_0}{E_0+\epsilon}(-G-(G^2+K_1)x)ds\\
				&= \frac{1}{1+\epsilon/E_0}(-G-(G^2+K_1)x)ds\\
	\end{align*}
	
	Como $\epsilon/E_0$ é bem pequeno, pode-se considerar que o termo $\frac{1}{1+\epsilon/E_0}$ é a soma de uma série geométrica. Logo, pode-se expandir este termo em um somatório:
	\begin{align*}
		d\theta &= \left(1 - \frac{\epsilon}{E_0} + \frac{\epsilon^2}{E_0^2}+ ...\right)(-G-(G^2+K_1)x)ds
	\end{align*}
	
	Novamente, mantendo apenas termos de primeira ordem,
	\begin{align*}
		d\theta &= \left(1 - \frac{\epsilon}{E_0}\right)(-G-(G^2+K_1)x)ds\\
				&= (-G-(G^2+K_1)x)ds + \left(G\left(\frac{\epsilon}{E_0}\right)+(G^2+K_1)x\left(\frac{\epsilon}{E_0}\right)\right)ds
	\end{align*}
	
	Aplicando novamente o argumento acima,
	\begin{align*}
		d\theta = \left\{-G-(G^2+K_1)x + G\frac{\epsilon}{E_0}\right\}ds
	\end{align*}
	
	Agora, pela equação \eqref{eq:2.12},
	\begin{align*}
		x'' &= \frac{d(\theta-\theta_0)}{ds}\\
			&= \frac{\left\{-G-(G^2+K_1)x + G\frac{\epsilon}{E_0}\right\}ds - (-G)ds}{ds}\\
			&= -(G^2+K_1)x + G\frac{\epsilon}{E_0}
	\end{align*}
\end{proof}

A equação correspondente para o movimento vertical é
\begin{align}
	z'' = K_1 z
\end{align}

\begin{proof}
	Para $z''$, pela força de Lorentz,
	\begin{align*}
		d\theta &= -\frac{d\ell}{\varrho} = + \frac{ecB}{E}d\ell\\
				&= \frac{ecB}{E}d\ell\\
				&= \frac{ecB}{E}(1+Gx)ds\\
				&= K_1 z (1+Gx)ds\\
				& = (K_1 z + GK_1xz)ds
	\end{align*}
	
	Descartando o termo de segunda ordem,
	\begin{align*}
		d\theta &= K_1 z ds\\
	\end{align*}
	
	Agora, pela equação \eqref{eq:2.12},
	\begin{align*}
		z'' &= \frac{d(\theta-\theta_0)}{ds}\\
			&= \frac{K_1 z\ ds}{ds}\\
			&= K_1 z
	\end{align*}
	lembrando que $\frac{d\theta_0}{ds} = 0$ porque não há componente vertical nesta variação.
\end{proof}

Definindo as funções focalizadoras $K_x$ e $K_z$ como
\begin{align}
	K_x(s) &= G^2(s)+K_1(s)\label{eq:2.21}\\
	K_z(s) &= - K_1(s)
\end{align}
tem-se que $x''$ e $z''$ podem ser descritos como
\begin{align}
	x'' &= -K_x(s)x + G(s)\frac{\epsilon}{E_0}\label{eq:2.19}\\
	z'' &= -K_z(s)z\label{eq:2.20}
\end{align}

O termo correspondente à $G^2$ está "faltando" de $K_z$ devido à consideração de que a órbita ideal está no plano, ou seja, não possui componente vertical. Mais especificamente, $G^2$ é um termo de força centrífuga, e um termo correspondente apareceria no movimento vertical se a órbita tivesse picos e vales. Anéis de armazenamento possuem, em geral, uma forte focalização. Neste caso, $G^2$ é bem menor que $K_1$, então $K_x$ e $K_z$ são aproximadamente iguais e possuem sinais opostos. Fisicamente, esta diferença de sinal significa que um elemento focalizador que focaliza em $x$ automaticamente desfocaliza em $z$, e vice-versa.

A equação de movimento em $z$ parece a equação de uma oscilação clássica (força proporcional ao desvio), com um coeficiente de força restauradora variável -- a função $K_z(s)$. A equação em $x$ é similar, exceto pela adição de um termo variável proporcional ao desvio de energia $\epsilon$. Nos campos guia que podem ser efetivamente utilizados, as soluções destas equações são, na verdade, oscilatórias, e descrevem oscilações laterais -- incluindo as chamadas oscilações betatron -- na trajetória do elétron. Estas oscilações são resultado das propriedades focalizadoras do campo guia, as quais caracterizam as funções de focalização $K_x$ e $K_z$.

Tanto $K_x$ quanto $K_z$ são funções periódicas ao longo do anel, logo
\begin{align}
	\begin{array}{rcl}
	\ K(s+L) & = & K(s)
	\end{array}
\end{align}

Por conveniência na construção e no design do anel, este possui uma periodicidade intrínseca. Ou seja, é composta por uma sequência de células magnéticas idênticas, cada célula sendo constituída por dipolos e quadrupolos. Então, para um anel com células de comprimento $\ell_c$,
\begin{align}
	\left\{\begin{array}{rcl}
	\ G(s+\ell_c) & = & G(s)\\
	K(s+\ell_c) & = & K_1(s)
	\end{array}\right.
\end{align}

Nota-se que, diferentemente da primeira propriedade, a periodicidade de célula é uma propriedade do design da máquina, não sendo totalmente verdadeira em campos reais devido a imperfeições na construção.

Na \autoref{fig:fig10}, a natureza das funções de focalização em uma parte do anel, abrangendo duas células. A Figura \ref{fig:fig10} (a) mostra o configuração dos dipolos e quadrupolos do anel. Os dipolos são denominados por B e tem um campo uniforme ($dB/dx = 0$). Os quadrupolos não tem campo na órbita ideal ($B_0=0$) e são denominados por F e D (F para focalizador e D para desfocalizador, ambos com relação ao movimento radial). As Figuras \ref{fig:fig10} (b) e (c) são as funções de focalização $G$, $K_x$ e $K_z$.

\begin{figure}[!htb]
	\centering
	\includegraphics[width=0.7\linewidth]{./Figuras/fig10.jpeg}
	\caption{Laço magnético e funções de focalização em uma célula de um campo guia em particular. Retirado de \cite{sands1970physics}.}
	\label{fig:fig10}
\end{figure}
    \subsection{Separação do movimento radial}
É conveniente separar o movimento radial em duas partes: uma parte sendo uma curva fechada deslocada da órbita de design -- a órbita de equilíbrio dos elétrons com desvio de energia -- e a outra parte sendo a oscilação transversal em torno desta órbita. Suponha que $x$ seja
	
\begin{align}
	x = x_{\epsilon} + x_{\beta}\label{eq:2.25}
\end{align}
então certamente a equação \eqref{eq:2.19} é satisfeita se as equações
\begin{align}
	x_\epsilon'' &= K_x(s)x_\epsilon + G(s)\frac{\epsilon}{E_0}\label{eq:2.26}\\
	x_\beta'' &= K_x(s)x_\beta\label{eq:2.27}
\end{align}
forem verdadeiras.
	
\begin{proof}
    Pela equação \eqref{eq:2.25}, $x = x_{\epsilon} + x_{\beta}$. Logo,
    \begin{align*}
        x'' &= x_{\epsilon}'' + x_{\beta}''\\
            &= K_x(s)x_\epsilon + G(s)\frac{\epsilon}{E_0} + K_x(s)x_\beta\\
            &= K_x(s)(x_\epsilon + x_\beta) + G(s)\frac{\epsilon}{E_0}\\
            &= K_x(s)x + G(s)\frac{\epsilon}{E_0}
    \end{align*}
\end{proof}
	
Definindo que $x_\epsilon(s)$ é uma função periódica em $s$ com período $L$, então $x_\epsilon(s)$ é a órbita fechada de um elétron com energia $E_0 + \epsilon$ (com $x_\beta=0$), e o movimento radial será a soma do desvio dessa nova órbita de equilíbrio e uma oscilação betatron.
	
O desvio $x_\epsilon$ é proporcional ao desvio de energia $\epsilon$. Define-se	
\begin{align}
	x_\epsilon(s) = \eta(s)\frac{\epsilon}{E_0}\label{eq:2.29}
\end{align}
onde $\eta(s)$ é a função única que satisfaz	
\begin{align}
	\begin{cases}
		\eta'' = K_x(s)\eta + G(s), \\
        \eta(0) = \eta(L), \\
        \eta'(0) = \eta'(L).
    \end{cases}
\end{align}
	
\begin{proof}
	Seja $x_\epsilon(s)$ dado pela equação \eqref{eq:2.29}. Pela equação \eqref{eq:2.26},
	\begin{align*}
        x_\epsilon'' &= K_x(s)x_\epsilon + G(s)\frac{\epsilon}{E_0}\\
        \left(\eta(s)\frac{\epsilon}{E_0}\right)'' &= K_x(s)\eta(s)\frac{\epsilon}{E_0} + G(s)\frac{\epsilon}{E_0}\\
        \eta(s)''\frac{\epsilon}{E_0} &= K_x(s)\eta(s)\frac{\epsilon}{E_0} + G(s)\frac{\epsilon}{E_0}\\
        \eta'' &= K_x(s)\eta + G(s)
	\end{align*}
\end{proof}
	
Denomina-se $\eta(s)$ como sendo a função de órbita fechada, e é uma função que caracteriza o campo guia total do anel. Note que $\eta (s)$ é a solução particular da equação diferencial \eqref{eq:2.26} em que a função é periódica com período $L$ e não depende de características do elétron, apenas da ótica do anel.
    \subsection{Trajetórias betatron}\label{sec:2.5}
As equações \eqref{eq:2.20} e \eqref{eq:2.27} descrevem as oscilações betatron vertical e radial, respectivamente. Considerando as aproximações feitas, o movimento em cada coordenada é independente. Logo, como $K_x(s)$ e $K_z(s)$ tem a mesma forma matemática, toma-se a forma representativa
	
\begin{align}
	x'' = -K(s) \; x \label{eq:2.31}
\end{align}
	
Lembrando que $K_x(s)$ descreve a oscilação betatron radial de um elétron com energia nominal $E_0$ e $K_z(s)$ descreve o movimento vertical.
	
A função de focalização $K(s)$ é definida em cada coordenada $s$ pelo design do anel de armazenamento. Se a posição e a inclinação ($x$ e $x'$) são dadas em alguma coordenada $s$, os termos subsequentes podem ser obtidos integrando a equação \eqref{eq:2.31}. Porém, como o campo guia é construído de segmentos magnéticos e $K(s)$ pode ser considerada constante nestes intervalos, pode-se integrar $x$ e $x'$ para cada segmento e juntar estes resultados. Dependendo do valor de $K$, $x$ é dado por

\begin{align}
	\left\{\begin{array}{l}
	K>0: \ \ x = a\ \cos(\sqrt{K}s+b) \\
	K=0: \ \ x = as+b \\
	K<0: \ \ x = a\ \cosh(\sqrt{-K}s+b)
	\end{array}\right. \label{eq:2.32}
\end{align}
onde $a$ e $b$ são constantes em cada segmento e podem ser determinadas pelos valores de $x$ e $x'$ na entrada do segmento (como $K$ é finita em qualquer ponto, $x$ e $x'$ devem ser ambas contínuas em todo ponto -- em particular, na junção entre dois segmentos).
	
\begin{proof}
	A equação \eqref{eq:2.31} é uma equação diferencial homogênea de 2ª ordem. Considerando que a função $K$ é uma constante, supõe-se que a solução da EDO é uma exponencial, 		ou seja, da forma $x=e^{rs}$. Substituindo esta possível solução, tem-se
	\begin{align*}
		x'' + Kx &=0\\
		(e^{rs})'' + K(e^{rs}) &= 0\\
		r^2e^{rs} + Ke^{rs}&=0\\
		(r^2 + K)e^{rs} &=0\\
		e^{rs} \neq 0 \therefore r^2 + K &= 0
	\end{align*}
	
	A equação $r^2 + K = 0$ é a equação característica da EDO. Resolvendo-a, tem-se $r = \pm \sqrt{-K}$, então $x_{1,2} = e^{\pm\sqrt{-K}s}$. Agora, existem 3 casos possíveis:
	\begin{itemize}
		\item $K>0$\\
			
		Com $K<0$, as raízes da equação característica são imaginárias. A solução geral da EDO é, para este caso, $x = c_1e^{\alpha s}\cos(\beta s) + c_2e^{\alpha s}\sin(\beta s)$. Como a função seno é apenas a função cosseno com uma diferença de fase, pode-se escrever
        \begin{align*}
        	x = a \ \cos(\sqrt{K}+b)
        \end{align*}
			
		\item $K = 0$\\
		
		A solução geral da EDO é, para este caso, $x = c_1e^{\sqrt{K}s} + sc_2e^{-\sqrt{K}s}$. Como $K=0$, 
		\begin{align*}
            x &= c_1e^{\sqrt{K}s} + sc_2e^{\sqrt{K}s}\\
            x &= c_1e^{0s} + sc_2e^{0s}\\
            x &= c_1 + sc_2
		\end{align*}
		
		Renomeando as constantes, $x = as + b$.

		\item $K < 0$\\
				
		A solução geral da EDO é, para este caso, $x = c_1e^{\sqrt{-K}s} + c_2e^{-\sqrt{-K}s}$. Fazendo $c_1=c_2=\frac{1}{2}$,
		\begin{align*}
            x &= c_1e^{\sqrt{-K}s} + c_2e^{-\sqrt{-K}s}\\
            x &= \frac{e^{\sqrt{-K}s} + e^{-\sqrt{-K}s}}{2} = cosh(\sqrt{-K}x)
		\end{align*}
				
		Para constantes $c_1$ e $c_2$ arbitrárias,
		\begin{align*}
          x = a \ cosh(\sqrt{-K}x + b)
		\end{align*}
	\end{itemize}
	
	Concluindo,
	\begin{align*}
        \left\{\begin{array}{l}
        K>0: \ \ x = a\ \cos(\sqrt{K}s+b) \\
        K=0: \ \ x = as+b \\
        K<0: \ \ x = a\ \cosh(\sqrt{-K}s+b)
        \end{array}\right.
	\end{align*}
\end{proof}

Como ilustração, supõe-se uma funçãoi $K(s)$ como a função $K_x(s)$ na \autoref{fig:fig10}. Duas possíveis trajetórias estão representadas na \autoref{fig:fig11} (b). A primeira é a trajetória que começa em $s_0$ com deslocamento unitário $x_0=1$ e nenhuma inclinação $x_0'=0$. A segunda começa com deslocamento nulo $x_0=0$ e inclinação unitária $x_0'= 1$. A primeira é chamada de \textit{cosinelike trajectory} -- $C$ e a segunda de \textit{sinelike trajectory} -- $S$. Seus detalhes dependem da coordenada de referência $s_0$ e são, em geral, funções não periódicas, mesmo que $K(s)$ seja. Para um anel com trajetórias estáveis, $C$ e $S$ são funções oscilatórias limitadas as quais possuem uma forma diferente a cada revolução.

\begin{figure}[!htb]
	\centering
	\includegraphics[width=0.7\linewidth]{./Figuras/fig11.jpeg}
	\caption{Função de focalização $K(s)$ e duas trajetórias: a \textit{cosine-like trajectory} e a \textit{sine-like trajectory} para uma coordenada de início $s_0 $. Retirado de \cite{sands1970physics}.}
	\label{fig:fig11}
\end{figure}
	
Agora, como a equação \eqref{eq:2.31} é linear em $x$, qualquer combinação linear de $C$ e $S$ também descreve uma trajetória possível para $x$. Mais que isso, qualquer trajetória pode ser descrita por esta combinação linear. Ou seja,
	
\begin{align}
	x(s) &= C(s,s_0)x_0 + S(s,s_0)x_0'\\
	x'(s) &= C'(s,s_0)x_0 + S'(s,s_0)x_0'
\end{align}
onde $C'$ e $S'$ são as derivadas de $C$ e $S$ em relação a $s$ e $x_0$ e $x_0'$ são os valores de $x$ e $x'$ em $s_0$. É conveniente escrever esta equação na forma matricial:
	
\begin{align}
	\boldsymbol{x}(s)=\boldsymbol{M}(s,s_0)\boldsymbol{x}(s_0)
\end{align}
onde 
\begin{align}
	\boldsymbol{x}(s) = \begin{bmatrix}
	x(s)\\ 
	x'(s)
	\end{bmatrix}
\end{align}
e
\begin{align}
	\boldsymbol{M}(s,s_0) = \begin{bmatrix}
	C(s,s_0) & S(s,s_0)\\
	C'(s,s_0) & S'(s,s_0)
	\end{bmatrix}\label{eq:2.37}
\end{align}
	
$\boldsymbol{M}(s,s_0)$ é a matriz de transferência de $s_0$ para $s$, a qual depende apenas de propriedades do campo guia entre duas coordenadas. A matriz de transferência de um trecho pode ser obtida em termos das matrizes de segmentos deste trecho. Logo, para um $s_1$ entre $s$ e $s_0$,
	
\begin{align}
	\boldsymbol{M}(s,s_0) = \boldsymbol{M}(s,s_1)\boldsymbol{M}(s_1,s_0)
\end{align}
	
\begin{proof}
	Pela definição da equação \eqref{eq:2.37},
	\begin{align*}
        \boldsymbol{M}(s,s_1) = \begin{bmatrix}
        C(s,s_1) & S(s,s_1)\\
        C'(s,s_1) & S'(s,s_1)
        \end{bmatrix}
	\end{align*} e
    \begin{align*}
		\boldsymbol{M}(s_1,s_0) = \begin{bmatrix}
		C(s_1,s_0) & S(s_1,s_0)\\
		C'(s_1,s_0) & S'(s_1,s_0)
		\end{bmatrix}
	\end{align*}
	
	Logo,
	\begin{align*}
        \boldsymbol{M}(s,s_0) &= \boldsymbol{M}(s,s_1)\boldsymbol{M}(s_1,s_0)\\
        &= \begin{bmatrix}
            C(s,s_1) & S(s,s_1)\\
            C'(s,s_1) & S'(s,s_1)
            \end{bmatrix} \begin{bmatrix}
                               C(s_1,s_0) & S(s_1,s_0)\\
                              C'(s_1,s_0) & S'(s_1,s_0)
                              \end{bmatrix}\\
        &= \begin{bmatrix}
        C(s,s_1)C(s_1,s_0)+S(s,s_1)C'(s_1,s_0) & C(s,s_1)S(s_1,s_0) + S(s,s_1)S'(s_1,s_0)\\
        C'(s,s_1)C(s_1,s_0)+S'(s,s_1)C'(s_1,s_0) & C'(s,s_1)S(s_1,s_0) + S'(s,s_1)S'(s_1,s_0)
        \end{bmatrix}	
	\end{align*}
	
	Considerando que $s_1 = s+\Delta_1$ e $s_0 = s_1 + \Delta_0$, então $(s,s_1) = \Delta_1$ e $(s_1,s_0) = \Delta_0$. Como $s_1$ é um ponto entre $s$ e $s_0$, também tem-se que $(s,s_0) = \Delta_1 + \Delta_0$. Logo, 
	\begin{align*}
        \boldsymbol{M}(\Delta_1)\boldsymbol{M}(\Delta_0) &= \begin{bmatrix}
        C(\Delta_1)C(\Delta_0)+S(\Delta_1)C'(\Delta_0) & C(\Delta_1)S(\Delta_0) + S(\Delta_1)S'(\Delta_0)\\
        C'(\Delta_1)C(\Delta_0)+S'(\Delta_1)C'(\Delta_0) & C'(\Delta_1)S(\Delta_0) + S'(\Delta_1)S'(\Delta_0)
        \end{bmatrix}
	\end{align*}
	
	Sabendo que $S'=C$ e $C'=-S$,
	\begin{align*}
        \boldsymbol{M}(\Delta_1)\boldsymbol{M}(\Delta_0) &= \begin{bmatrix}
        C(\Delta_1)C(\Delta_0)-S(\Delta_1)S(\Delta_0) & C(\Delta_1)S(\Delta_0) + S(\Delta_1)C(\Delta_0)\\
        -S(\Delta_1)C(\Delta_0)-C(\Delta_1)S(\Delta_0) & -S(\Delta_1)S(\Delta_0) + C(\Delta_1)C(\Delta_0)
        \end{bmatrix}
	\end{align*}
	
	Por propriedades trigonométricas,
	\begin{align*}
        [\boldsymbol{M}(\Delta_1)\boldsymbol{M}(\Delta_0)]_{11} &= \frac{1}{2}[C(\Delta_1-\Delta_0)+C(\Delta_1+\Delta_0)] - \frac{1}{2}[C(\Delta_1-\Delta_0)-C(\Delta_1+\Delta_0)]\\
        [\boldsymbol{M}(\Delta_1)\boldsymbol{M}(\Delta_0)]_{12} &= \frac{1}{2}[S(\Delta_1+\Delta_0)-S(\Delta_1-\Delta_0)] + \frac{1}{2}[S(\Delta_1-\Delta_0)+S(\Delta_1+\Delta_0)]\\
        [\boldsymbol{M}(\Delta_1)\boldsymbol{M}(\Delta_0)]_{21} &= -\frac{1}{2}[S(\Delta_1-\Delta_0)+S(\Delta_1+\Delta_0)] - \frac{1}{2}[S(\Delta_1+\Delta_0)-S(\Delta_1-\Delta_0)]\\
        [\boldsymbol{M}(\Delta_1)\boldsymbol{M}(\Delta_0)]_{22} &= -\frac{1}{2}[C(\Delta_1-\Delta_0)-C(\Delta_1+\Delta_0)] + \frac{1}{2}[C(\Delta_1-\Delta_0)+C(\Delta_1+\Delta_0)]
	\end{align*}
	
	Simplificando os termos da matriz, tem-se
	\begin{align*}
        \boldsymbol{M}(\Delta_1)\boldsymbol{M}(\Delta_0) &= \begin{bmatrix}
        C(\Delta_1+\Delta_0) & S(\Delta_1+\Delta_0)\\
        -S(\Delta_1+\Delta_0) & C(\Delta_1+\Delta_0)
        \end{bmatrix}\\
        &= \begin{bmatrix}
            C(\Delta_1+\Delta_0) & S(\Delta_1+\Delta_0)\\
            C'(\Delta_1+\Delta_0) & S'(\Delta_1+\Delta_0)
            \end{bmatrix}\\
        \boldsymbol{M}(s,s_1)\boldsymbol{M}(s_1,s_0)&= \begin{bmatrix}
            C(s,s_0) & S(s,s_0)\\
            C'(s,s_0) & S'(s,s_0)
            \end{bmatrix} = \boldsymbol{M}(s,s_0) 
	\end{align*}
	c.q.d.
\end{proof}
	
Para os três casos possíveis de $K$ descritos na equação \eqref{eq:2.32},
	
\begin{align}
	K>0: \ \ \boldsymbol{M}(s_2,s_1) &= \begin{bmatrix}
	\cos(\sqrt{K}\ell) & \frac{1}{\sqrt{K}}\ sen(\sqrt{K}\ell)\\
	-\sqrt{K}\ sen(\sqrt{K}\ell) & \cos(\sqrt{K}\ell)
	\end{bmatrix}\\
	K=0: \ \ \boldsymbol{M}(s_2,s_1) &= \begin{bmatrix}
		1 & \ell\\
		0 & 1
		\end{bmatrix}\\
	K<0: \ \ \boldsymbol{M}(s_2,s_1) &= \begin{bmatrix}
		\cosh(\sqrt{-K}\ell) & \frac{1}{\sqrt{-K}}\ senh(\sqrt{-K}\ell)\\
		\sqrt{-K}\sinh(\sqrt{-K}\ell) & \cosh(\sqrt{-K}\ell)
		\end{bmatrix}
\end{align}
onde $\ell = s_2-s_1$.
	
\begin{proof}Analisando para os três casos de $K$:
	\begin{itemize}
	\item $K>0$\\
	
	Da equação \eqref{eq:2.32}, tem-se que $C(s_2,s_1) = \cos(\sqrt{K}\ell)$. Logo,
	\begin{align*}
        C'(s_2,s_1) &= \frac{d\ \cos(\sqrt{K}\ell)}{d\ell} = -\sqrt{K}\sin(\sqrt{K}\ell)\\
        S(s_2,s_1) &= \int \cos(\sqrt{K}\ell)d\ell = \frac{1}{\sqrt{K}}\sin(\sqrt{K}\ell)\\
        \therefore \boldsymbol{M}(s_2,s_1) &= \begin{bmatrix}
            \cos(\sqrt{K}\ell) & \frac{1}{\sqrt{K}}\ sen(\sqrt{K}\ell)\\
            -\sqrt{K}\ sen(\sqrt{K}\ell) & \cos(\sqrt{K}\ell)
            \end{bmatrix}
	\end{align*}
	\item $K=0$\\
	
	Da equação \eqref{eq:2.32}, tem-se que $C(s_2,s_1) = 1$. Logo,
	\begin{align*}
		C'(s_2,s_1) &= \frac{d \ 1}{d\ell} = 0\\
		S(s_2,s_1) &= \int d\ell = \ell\\
		\therefore \boldsymbol{M}(s_2,s_1) &= \begin{bmatrix}
				1 & \ell\\
				0 & 1
				\end{bmatrix}
	\end{align*}
	\item $K<0$\\
	
	Da equação \eqref{eq:2.32}, tem-se que $C(s_2,s_1) = \cosh(\sqrt{-K}\ell)$. Logo,
		\begin{align*}
        	C'(s_2,s_1) &= \frac{d\ \cosh(\sqrt{-K}\ell)}{d\ell} = \sqrt{-K}senh(\sqrt{-K}\ell)\\
            S(s_2,s_1) &= \int \cosh(\sqrt{-K}\ell)d\ell = \frac{1}{\sqrt{-K}}senh(\sqrt{-K}\ell)\\
            \therefore \boldsymbol{M}(s_2,s_1) &= \begin{bmatrix}
            \cosh(\sqrt{-K}\ell) & \frac{1}{\sqrt{-K}}\ senh(\sqrt{-K}\ell)\\
            \sqrt{-K}\ senh(\sqrt{-K}\ell) & \cosh(\sqrt{-K}\ell)
            \end{bmatrix}
		\end{align*}
	\end{itemize}
\end{proof}
	
A solução geral da equação \eqref{eq:2.31} pode ser escrita como
	
\begin{align}
	x(s) = a\zeta(s)\ cos\{\varphi(s)-\vartheta\}\label{eq:2.39}
\end{align}
onde $\zeta(s)$ e $\varphi(s)$ são funções especialmente definidas em $s$ com certas propriedades convenientes, e $a$ e $\vartheta$ são constantes obtidas pelas condições iniciais, as quais determinam uma trajetória particular. Define-se
	
\begin{align}
	\varphi(s) = \int_{0}^{s} \frac{d\bar{s}}{\zeta^2(\bar{s})}
\end{align}
	
Então
\begin{align}
	\varphi'(s) = \frac{1}{\zeta^2}
\end{align}
e, se $\zeta(s)$ for definida para ser essa função positiva, analítica que satisfaz
\begin{align}
	\zeta'' = -K(s)\zeta+\frac{1}{\zeta^3}\label{eq:2.41}
\end{align}
então $x(s)$ da equação \eqref{eq:2.39} satisfaz a equação diferencial \eqref{eq:2.31}.
	
\begin{proof}
	Seja $x(s)$ dado pela equação \eqref{eq:2.39}. Então,
	\begin{align*}
        x' &= [a\zeta\ \cos\{\varphi-\vartheta\}]'\\
        &= a[\zeta'\cos(\varphi-\vartheta)-\zeta\varphi'\sin(\varphi-\vartheta)]\\
        \therefore x'' &= a[\zeta'\cos(\varphi-\vartheta)-\zeta\varphi'\sin(\varphi-\vartheta)]'\\
        &= a\left[\zeta''\cos(\varphi-\vartheta)-\zeta'\varphi'\sin(\varphi-\vartheta) - \left(-\frac{\zeta'}{\zeta^2}\sin(\varphi-\vartheta)+\frac{\varphi'}{\zeta}\cos(\varphi-\vartheta)\right)\right]
	\end{align*}
	
	Pela equação \eqref{eq:2.41}, 
	\begin{align*}
        x'' &= a\left[\left(-K\zeta+\frac{1}{\zeta^3}\right)\cos(\varphi-\vartheta)-\frac{\zeta'}{\zeta^2}\sin(\varphi-\vartheta) + \frac{\zeta'}{\zeta^2}\sin(\varphi-\vartheta)-\frac{1}{\zeta^3}\cos(\varphi-\vartheta)\right]\\
        &= -Ka\zeta\ \cos(\varphi-\vartheta)\\
        &= -Kx
	\end{align*}
	
	Assim, pode-se ver que $x(s) = a\zeta(s)\ \cos\{\varphi(s)-\vartheta\}$ é solução da equação diferencial \eqref{eq:2.31}.
\end{proof}
	
Tradicionalmente, define-se a função betatron $\beta(s)$ como
\begin{align}
	\beta(s) = \zeta^2(s)
\end{align}
então
\begin{align}
	x(s) &= a\sqrt{\beta(s)}\ \cos\{\varphi(s)-\vartheta\}\label{eq:2.43}\\
	\varphi(s) &= \int_{0}^{s} \frac{d\bar{s}}{\beta(\bar{s})}\label{eq:2.44}
\end{align}
	
Note que, dada a função de focalização $K(s)$ do anel, $\beta(s)$ pode ser unicamente determinada. Porém, enquanto $K(s)$ é dada em função das propriedades locais do campo guia, a função $\beta(s)$ -- ou $\zeta(s)$ depende da configuração total do anel. Por outro lado, uma vez que $\beta(s)$ é conhecida, $K(s)$ pode ser imediatamente determinada pelas suas derivadas locais, mas é $\beta(s)$ que revela de forma mais direta as características significantes da trajetória dos elétrons armazenados.
	
Lembrando que toda a discussão feita nesta subsseção se aplica tanto no movimento radial quanto no vertical, ou seja, o anel é descrito pelas funções $\beta_x$ e $\beta_z$, as quais são derivadas das funções de focalização $K_x$ e $K_z$, respectivamente.
    \subsection{Oscilações betatron pseudo-harmônicas}
As equações \eqref{eq:2.43} e \eqref{eq:2.44} descrevem completamente o caminho realizado pelo elétron. Para ter uma visão completa do movimento do elétron, basta apenas adicionar o fato de que o elétron viaja sempre na velocidade $c$ da luz. Fazendo uma aproximação, é adequado tomar que a coordenada longitudinal $s$ varia simplesmente como
	
\begin{align}
	s = s_0 + ct\label{eq:2.48}
\end{align}
	
A forma correta da equação \eqref{eq:2.48} será discutida na Seção \ref{sec:3.2}.
	
A função betatron descreve completamente as propriedades laterais de focalização do campo guia. Pela sua natureza, a função betatron deve ser sempre positiva definida, e sua curva tem uma forma semelhante a uma onda. Ela também é periódica ao longo do anel, logo
	
\begin{align}
	\beta(s+L) = \beta(s)
\end{align}
	
A função betatron possui um valor único em cada coordenada $s$. Se o campo guia é dividido em células idênticas (desconsiderando imperfeições de construção), $\beta$ terá a mesma simetria.
	
Conforme o elétron viaja ao redor do anel, ele executa uma oscilação lateral que não é nem harmônica e nem periódica. O movimento é um tipo de onda senoidal (\ref{fig:fig12}) distorcida com uma amplitude $a\sqrt{\beta}$ variante, a qual é modulada proporcionalmente à raiz da função betatron e com uma fase $(\varphi-\upsilon)$ que avança com $s$ a uma taxa de variação proporcional a $\frac{1}{\beta}$.
	
\begin{figure}[!htb]
	\centering
	\includegraphics[width=0.7\linewidth]{./Figuras/fig12.jpeg}
	\caption{(a) Função betatron. (b) \textit{Cosine-like trajectory} para $s=0$. (c) \textit{Sine-like trajectory} para $s=0$. (d) Uma trajetória depois de várias revoluções sucessivas. Retirado de \cite{sands1970physics}.}
	\label{fig:fig12}
\end{figure}
	
Uma importante propriedade do movimento betatron é evidente na \ref{fig:fig12}(d) -- em cada coordenada, o desvio $x$ de um elétron em movimento fica sempre abaixo de um valor limitante $X(s)$, o qual é obtido colocando $cos(\varphi-\upsilon)=1$, ou seja,
	
\begin{align}
	X(s) = a\sqrt{\beta(s)}
\end{align}
	
A trajetória completa de um elétron armazenado cairá sempre dentro de um envelope definido por $\pm X(s)$. Segue que a abertura necessária para conter um elétron com uma amplitude de oscilação que varia ao redor do anel varia como $X(s)$. A relação entre a largura do envelope em duas coordenadas $s_1$ e $s_2$ é
	
\begin{align}
	\frac{X_2}{X_1} = \sqrt{\frac{\beta_2}{\beta_1}}
\end{align}
	
Para analisar a inclinação da trajetória betatron, ou seja, $x'=\frac{dx}{ds}$, considera-se a derivada da equação \eqref{eq:2.43}:
	
\begin{align}
	x' = - \frac{a}{\sqrt{\beta}}sen(\varphi-\upsilon)+\frac{\beta'}{2\beta}x\label{eq:2.52}
\end{align}
	
O primeiro termo vem da mudança de fase, e o segundo da variação de $\beta$.
	
\begin{proof}
	Seja $x(s) = a\sqrt{\beta(s)}\ cos\{\varphi(s)-\upsilon\}$. Logo, 
	\begin{align*}
      x' &= [a\sqrt{\beta}\ cos\{\varphi-\upsilon\}]'\\
         &= a[(\sqrt{\beta})'cos(\varphi-\upsilon)-\sqrt{\beta}\varphi'sen(\varphi-\upsilon)]\\
      	 &= a\left[\frac{\beta'}{2\sqrt{\beta}}cos(\varphi-\upsilon)-\sqrt{\beta}\frac{1}{\beta}sen(\varphi-\upsilon)\right]\\
      	 &= a\left[\frac{\beta'}{2\sqrt{\beta}}cos(\varphi-\upsilon)-\frac{1}{\sqrt{\beta}}sen(\varphi-\upsilon)\right]\\
      	 &= \frac{\beta'}{2\sqrt{\beta}}a\ cos(\varphi-\upsilon)-\frac{a}{\sqrt{\beta}}sen(\varphi-\upsilon)\\
      	 &= \frac{\beta'}{2\beta}a\sqrt{\beta}\ cos(\varphi-\upsilon)-\frac{a}{\sqrt{\beta}}sen(\varphi-\upsilon)\\
      	 &= -\frac{a}{\sqrt{\beta}}sen(\varphi-\upsilon)+\frac{\beta'}{2\beta}x
	\end{align*}
\end{proof}
	
Note que os zeros de $x'$ -- e, portanto, os valores de pico de $x$ -- não ocorrem em $cos(\varphi-\upsilon)=1$. Eles ocorrem em
	
\begin{align}
	tg(\varphi-\upsilon) = \frac{\beta'}{2}
\end{align}
o que significa que
\begin{align}
	cos(\varphi-\upsilon) = \left[1+\frac{\beta'^2}{4}\right]^{-\frac{1}{2}}
\end{align}
	
\begin{proof}
	Fazendo $x'=0$:
	\begin{align*}
        x'&=0\\
        -\frac{a}{\sqrt{\beta}}sen(\varphi-\upsilon)+\frac{\beta'}{2\beta}x &= 0\\
        \frac{\beta'}{2\beta}x &= \frac{a}{\sqrt{\beta}}sen(\varphi-\upsilon)\\
        \frac{\beta'}{2\beta}a\sqrt{\beta}\ cos\{\varphi-\upsilon\} &= \frac{a}{\sqrt{\beta}}sen(\varphi-\upsilon)\\
        \frac{\beta'}{2\beta}\sqrt{\beta}\sqrt{\beta} &= \frac{a}{a}\frac{sen(\varphi-\upsilon)}{cos(\varphi-\upsilon)}\\
        \frac{\beta'}{2} &= tg(\varphi-\upsilon)
	\end{align*}
	
	Pela identidade trigonométrica $1+tg^2(x) = sec^2(x)$,
	\begin{align*}
        1+tg^2(\varphi-\upsilon) &= sec^2(\varphi-\upsilon)\\
        1+\left(\frac{\beta'}{2}\right)^2 &= \left(\frac{1}{cos(\varphi-\upsilon)}\right)^2\\
        1+\frac{\beta'^2}{4} &= \frac{1}{cos^2(\varphi-\upsilon)}\\
        cos^2(\varphi-\upsilon) &= \left[1+\frac{\beta'^2}{4}\right]^{-1}\\
        cos(\varphi-\upsilon) &= \left[1+\frac{\beta'^2}{4}\right]^{-\frac{1}{2}}
	\end{align*}
	c.q.d.
\end{proof}
	
Se o pico de um ciclo particular de uma oscilação ocorrer em algum $s$, o valor de pico do desvio será
	
\begin{align}
	x_{pico} = a\sqrt{\beta}\left[1+\frac{\beta'^2}{4}\right]^{-\frac{1}{2}}
\end{align}

Veja a \autoref{fig:fig13}.

\begin{figure}[!htb]
	\centering
	\includegraphics[width=0.8\linewidth]{./Figuras/fig13.jpeg}
	\caption{O máximo de um ciclo particular de uma oscilação betatron. Retirado de \cite{sands1970physics}.}
	\label{fig:fig13}
\end{figure}
	
Em uma oscilação harmônica clássica, a amplitude é uma invariante do movimento. Seu quadrado é proporcional à energia da oscilação, e pode ser expresso como uma função quadrática da posição e velocidade instantâneas. O invariante correspondente do oscilador pseudo-harmônico é a constante $a$, e esta pode ser obtida em termos de $x$ e $x'$ pela equação
	
\begin{align}
	a^2 = \frac{x^2}{\beta} + \beta\left[x'-\frac{\beta'}{2\beta}x\right]^2\label{eq:2.56}
\end{align}
	
\begin{proof}
	Pela equação \eqref{eq:2.43},
	\begin{align*}
        x &= a\sqrt{\beta}\ cos\{\varphi-\upsilon\}\\
        \frac{x}{a\sqrt{\beta}} &= cos(\varphi-\upsilon)\\
        \left(\frac{x}{a\sqrt{\beta}}\right)^2 &= cos^2(\varphi-\upsilon)\\
        \frac{x^2}{a^2\beta} &= cos^2(\varphi-\upsilon)
	\end{align*}
	
	Já pela equação \eqref{eq:2.52},
	\begin{align*}
        x' &= - \frac{a}{\sqrt{\beta}}sen(\varphi-\upsilon)+\frac{\beta'}{2\beta}x\\
        x' - \frac{\beta'}{2\beta}x&= - \frac{a}{\sqrt{\beta}}sen(\varphi-\upsilon)\\
        \frac{\sqrt{\beta}}{a}\left[x' - \frac{\beta'}{2\beta}x\right]&= -sen(\varphi-\upsilon)\\
        \left(\frac{\sqrt{\beta}}{a}\left[x' - \frac{\beta'}{2\beta}x\right]\right)^2&= sen^2(\varphi-\upsilon)\\
        \frac{\beta}{a^2}\left[x' - \frac{\beta'}{2\beta}x\right]^2&= sen^2(\varphi-\upsilon)
	\end{align*}
	
	Pela relação trigonométrica $sen^2(x)+cos^2(x)=1$,
	\begin{align*}
        sen^2(\varphi-\upsilon)+cos^2(\varphi-\upsilon)&=1\\
        \frac{\beta}{a^2}\left[x' - \frac{\beta'}{2\beta}x\right]^2 + \frac{x^2}{a^2\beta} &= 1\\
        \frac{1}{a^2}\left(\beta\left[x' - \frac{\beta'}{2\beta}x\right]^2 + \frac{x^2}{\beta}\right) &= 1\\
        \beta\left[x' - \frac{\beta'}{2\beta}x\right]^2 + \frac{x^2}{\beta} &= a^2
	\end{align*}
	c.q.d.
\end{proof}
	
Se os valores de $x$ e $x'$ são conhecidos em alguma coordenada, supõe-se $s_1$, então a constante $a$ pode ser obtida e todos os valores subsequentes de $x$ e $x'$ podem ser expressos por
	
\begin{align}
	x = \frac{1}{\sqrt{\beta_1}}\left[x_1^2+\left(\beta_1x'_1-\frac{x_1\beta'_1}{2}\right)^2\right]^\frac{1}{2}\sqrt{\beta}\ cos(\varphi-\upsilon)\label{eq:2.57}
\end{align}
	
\begin{proof}
	Pela equação \eqref{eq:2.56},
	\begin{align*}
        a^2 &= \frac{x^2}{\beta} + \beta\left[x'-\frac{\beta'}{2\beta}x\right]^2\\
        	&= \frac{1}{\beta}\left(x^2 + \beta^2\left[x'-\frac{\beta'}{2\beta}x\right]^2\right)\\
        	&= \frac{1}{\beta}\left(x^2 + \left[\beta x'-\frac{\beta'}{2}x\right]^2\right)\\
        \therefore a &= \left[\frac{1}{\beta}\left(x^2 + \left[\beta x'-\frac{\beta'}{2}x\right]^2\right)\right]^\frac{1}{2}\\
        	&= \frac{1}{\sqrt{\beta}}\left(x^2 + \left[\beta x'-\frac{\beta'}{2}x\right]^2\right)^\frac{1}{2}
	\end{align*}
	
	Sejam $x(s_1)=x_1$, $x'(s_1)=x'_1$ e $\beta(s_1)=\beta_1$ os valores de $x$, $x'$ e $\beta$ conhecidos no ponto $s_1$, então $a$ pode ser determinado com estes valores:
	\begin{align*}
		a = \frac{1}{\sqrt{\beta_1}}\left(x_1^2 + \left[\beta_1 x_1'-\frac{\beta_1'}{2}x_1\right]^2\right)^\frac{1}{2}
	\end{align*}
	
	Substituindo o valor de $a$ na equação \eqref{eq:2.43},
	\begin{align*}
		x &= a\sqrt{\beta}\ cos\{\varphi-\upsilon\}\\
		  &= \frac{1}{\sqrt{\beta_1}}\left[x_1^2+\left(\beta_1x'_1-\frac{x_1\beta'_1}{2}\right)^2\right]^\frac{1}{2}\sqrt{\beta}\ cos(\varphi-\upsilon)
	\end{align*}
	c.q.d.
\end{proof}
	
A constante de fase $\upsilon$ também precisa ser determinada de $x$ e $x'$, e esta pode ser obtida pela equação
	
\begin{align}
	tg(\varphi_1 - \upsilon) = -\frac{\beta_1 x'_1}{x_1}+\frac{\beta'_1}{2}
\end{align}
onde $\varphi_1 = \varphi(s_1)$.
	
\begin{proof}
	Já foi deduzido anteriormente que
	\begin{align*}
        sen(\varphi-\upsilon) &= -\frac{\sqrt{\beta}}{a}\left[x' - \frac{\beta'}{2\beta}x\right]\\
        cos(\varphi-\upsilon) &= \frac{x}{a\sqrt{\beta}}
	\end{align*}
	
	Para obter $tg(\varphi-\upsilon)$, basta
	\begin{align*}
		tg(\varphi-\upsilon) &= \frac{sen(\varphi-\upsilon)}{cos(\varphi-\upsilon)}\\
							 &= \frac{-\frac{\sqrt{\beta}}{a}\left[x' - \frac{\beta'}{2\beta}x\right]}{\frac{x}{a\sqrt{\beta}}}\\
							 &= -\frac{\sqrt{\beta}}{a}\left[x' - \frac{\beta'}{2\beta}x\right] \frac{a\sqrt{\beta}}{x}\\
							 &= \frac{\beta}{x}\left[-x' + \frac{\beta'}{2\beta}x\right]\\
							 &= -\frac{\beta x'}{x} + \frac{\beta'}{2}\\
		\therefore tg(\varphi_1-\upsilon) &= -\frac{\beta_1  x_1'}{x_1} + \frac{\beta_1'}{2}
	\end{align*}
	
	Isolando $\upsilon$, pode-se obtê-lo diretamente pela relação
	\begin{align*}
		tg(\varphi_1-\upsilon) &= -\frac{\beta_1  x_1'}{x_1} + \frac{\beta_1'}{2}\\
		cotg(tg(\varphi_1-\upsilon)) &= cotg\left(-\frac{\beta_1  x_1'}{x_1} + \frac{\beta_1'}{2}\right)\\
		\varphi_1-\upsilon &= cotg\left(-\frac{\beta_1  x_1'}{x_1} + \frac{\beta_1'}{2}\right)\\
		\upsilon &= \varphi_1 - cotg\left(-\frac{\beta_1  x_1'}{x_1} + \frac{\beta_1'}{2}\right)
	\end{align*}
\end{proof}
	
Para obter o valor máximo $X(s)$ que pode ser alcançado em qualquer $s$ em qualquer revolução subsequente, basta substituir $cos(\varphi-\upsilon)=1$ na equação \eqref{eq:2.57}:
	
\begin{align}
	X(s) = \frac{1}{\sqrt{\beta_1}}\left[x_1^2+\left(\beta_1x'_1-\frac{x_1\beta'_1}{2}\right)^2\right]^\frac{1}{2}\sqrt{\beta(s)}
\end{align}
	
Note que $X(s)$ independe de $\upsilon$.
	
Geralmente, é esperado que as amplitudes resultantes de distúrbios na trajetória serão menores quanto menor for $\beta$. De fato, pode-se considerar que $\frac{1}{\beta}$ é uma medida da "força" da focalização lateral, e que pequenos valores de $\beta$ são normalmente desejáveis. 
    \subsection{Sintonias}
Conforme um elétron completa uma revolução dentro do anel de armazenamento em uma coordenada qualquer $s_0$, sua oscilação de fase $(\varphi - \vartheta)$ avança de
\begin{align}
	\int\limits_{s_0}^{s_0+L}\frac{ds}{\beta}
\end{align}

Devido à periodicidade de $\beta$, esta integral tem o mesmo valor para qualquer $s_0$. Ou seja, em uma revolução completa, a fase da trajetória do elétron é aumentada sempre do mesmo valor. Esse avanço de fase é um fator importante do anel de armazenamento, e é escrito normalmente como $2\pi \nu$, e $\nu$ é chamado de número betatron ou sintonia da máquina. Sua definição é
\begin{align}
	\nu = \frac{1}{2 \pi}\int\limits_{s}^{s+L}\frac{d\bar{s}}{\beta} = \frac{1}{2 \pi}\int\limits_{0}^{L}\frac{ds}{\beta} = \frac{1}{2\pi}\oint \frac{ds}{\beta}\label{eq:2.60}
\end{align}

(O símbolo $\oint$ irá indicar qualquer integração ao redor de todo o anel).
A sintonia é um parâmetro global da máquina, ou seja, não depende da coordenada $s$: ela é a mesma para todo o anel.

As sintonias das coordenadas $x$ e $z$ -- indicadas por $\nu_x$ e $\nu_z$ -- são geralmente diferentes, sendo derivadas das duas funções betatron $\beta_x$ e $\beta_z$. Tanto $\nu_x$ quanto $\nu_z$ são, tipicamente, números não muito grandes próximos de, mas não exatamente, um quarto de inteiro (2.78 ou 5.15, por exemplo). 

Apesar da trajetória betatron ser uma oscilação contorcida não-periódica, se observar uma coordenada fixa ao longo de sucessivas revoluções de um elétron, pode-se concluir que este desvio segue uma forma senoidal. Suponha que esta coordenada fixa é $s_0$ e as sucessivas passagens do elétron sejam indexadas por $j=0,1,2,3,\ ...$. Também considere $\varphi_0$ a fase na passagem 0. Nas passagens seguintes, a fase do movimento irá aumentar de $2\pi\nu$ e, na j-ésima passagem, a fase será
\begin{align}
	2\pi\nu j+ \varphi_0
\end{align}
e o desvio será
\begin{align}
	x_j = a\sqrt{\beta_0}\ cos(2\pi\nu j+ \varphi_0)\label{eq:2.61}
\end{align}
onde $\beta_0$ é o valor da função $\beta$ na coordenada $s_0$.

A amplitude $a\sqrt{\beta_0}$ é constante, então o desvio, avaliado a cada revolução, varia simplesmente como uma simples oscilação senoidal. Como o tempo de cada revolução é constante (desprezando uma pequena correção proporcional a $x$), dado por $\frac{L}{c}$, pode-se descrever o tempo $t_j$ da j-ésima passagem como
\begin{align}
	t_j = \frac{L}{c}j
\end{align}
ou que
\begin{align}
	2\pi j = \omega_r t_j
\end{align}
onde
\begin{align}
	\omega_r = 2\pi \frac{c}{L}
\end{align}
é a frequência angular de revolução do elétron. Então a equação \eqref{eq:2.61} pode ser reescrita, para qualquer $s$ fixo, como
\begin{align}
	x_s (t_j) = a\sqrt{\beta(s)}\ cos(\nu \omega_r t_j + \varphi_{0s})\label{eq:2.64}
\end{align}

\begin{proof}
	Pela equação \eqref{eq:2.48}, $s = s_0 + ct$. Logo, em uma revolução,
	\begin{align*}
		s-s_0 &= ct\\
		L &= ct\\
		\therefore t &= \frac{L}{c}
	\end{align*}
	
	Para a j-ésima revolução,
	\begin{align*}
		t_j = \frac{L}{c}j
	\end{align*}
	
	Isolando $j$,
	\begin{align*}
		j &= \frac{c}{L} t_j\\
		\therefore 2\pi j &= 2\pi \frac{c}{L} t_j
	\end{align*}
	
	Definindo $\omega_r = 2\pi \frac{c}{L}$, então
	\begin{align*}
		2\pi j = \omega_r t_j
	\end{align*}
	e, substituindo isto na equação \eqref{eq:2.61}, tem-se
	\begin{align*}
		x_j = a\sqrt{\beta_0}\ cos(\nu \omega_r t_j + \varphi_{0})
	\end{align*}
	
	Considerando o caso de qualquer coordenada $s$,
	\begin{align*}
		x_s (t_j) = a\sqrt{\beta(s)}\ cos(\nu \omega_r t_j + \varphi_{0s})
	\end{align*}
\end{proof}

Quando uma coordenada $s$ em particular é observada, o movimento lateral é indistinguível de uma simples oscilação harmônica na frequência $\nu \omega_r$ -- normalmente chamada de frequência betatron.

Observando a equação \eqref{eq:2.61}, pode-se ver a justificativa para a afirmativa feita na Seção anterior de que, em cada coordenada longitudinal, deve-se esperar que em algum momento $x$ assumirá seu valor máximo $X(s) = a\sqrt{\beta(s)}$. A menos que $\nu$ seja um número inteiro ou, ainda mais genérico, a não ser que a diferença entre $\nu$ e um inteiro seja uma fração simples -- o que torna a não ser exatamente verdade em um anel de armazenamento real -- a fase (de módulo $2\pi$) em sucessivas passagens de qualquer ponto fixo irá passar por um grande número de valores entre $0$ e $2\pi$ antes de se repetir. E o deslocamento irá em algum momento atingir seu valor de pico $X$ em cada coordenada.

O significado mais importante da sintonia $\nu$ da máquina é relacionado com a existência de ressonâncias que aparecem se $\nu$ assume certos valores. Por exemplo, se $\nu$ é um inteiro, a oscilação betatron iria, idealmente, tornar-se ligeiramente periódica -- repetindo-se a cada revolução. Entretanto, a menor imperfeição no campo guia irá agir como uma perturbação, a qual é síncrona com a frequência de oscilação. Uma pertubação síncrona leva a uma excitação ressonante da oscilação e a um crescimento exponencial da amplitude. Não haverá oscilação estável. Mais adiante será mostrado que outras ressonâncias ocorrem também quando $\nu$ é metade de um inteiro e, se efeitos não-lineares forem considerados, quando a diferença entre $\nu$ e um inteiro é qualquer fração simples.

Ressonâncias devem ser, claro, evitadas nas oscilações betatron radial e vertical. Tem-se que ressonâncias de algum tipo podem ocorrer quando $\nu_x$ e $\nu_z$ satisfazem
\begin{align}
	m \nu_x + n \nu_z = r\label{eq:2.65}
\end{align}
onde $m$, $n$ e $r$ são inteiros. Efeitos significativos são geralmente observados apenas em ressonâncias de ordem baixa, ou seja, estas em que $m$, $n$ e $r$ possuem valores baixos entre 0,1,2,3. O ponto de operação de um anel de armazenamento é especificado por $\nu_x$ e $\nu_z$ e deve ser escolhido de forma a evitar ressonâncias. A relação de ressonância \eqref{eq:2.65} define um conjunto de linhas em um diagrama $\nu_x$, $\nu_z$. Algumas delas estão representadas na \autoref{fig:fig14}, onde um possível ponto de operação também é indicado.

\begin{figure}[!htb]
	\centering
	\includegraphics[width=0.6\linewidth]{./Figuras/fig14.jpeg}
	\caption{Ressonâncias de ordem baixa em um diagrama $\nu_x$, $\nu_z$. Retirado de \cite{sands1970physics}.}
	\label{fig:fig14}
\end{figure}

Para um grupo particular de ressonâncias onde $\nu_x$ é igual a $\nu_z$ ou a diferença entre eles é um inteiro, haverá um forte acoplamento entre as oscilações horizontal e vertical. Nesta ressonância, a suposição de que as oscilações são completamente independentes não é mais válida e a modelagem do movimento dos elétrons fica mais complicada. Às vezes, um anel de armazenamento pode ser intencionalmente operadora em uma, ou perto de uma, ressonância de acoplamento a fim de aumentar a amplitude de oscilação vertical alimentando-a com a energia vinda da oscilação radial.

Para estar seguro de ressonâncias perigosas, é preciso que o ponto de operação real esteja perto o suficiente do ponto projetado -- como pode-se ver na \autoref{fig:fig14}. Espera-se que as imperfeições dos ímãs irão causar mudanças em $\nu$ proporcionais ao próprio $\nu$. Um anel de armazenamento com uma sintonia grande é propensa a ser uma máquina "sensível". Por este fato, a sintonia normalmente é escolhida entre valores de 2 a 6.
	\subsection{Descrição aproximada das oscilações betatron}
	\subsection{Natureza da função beta}
	\subsection{Perturbação de órbita fechada}
	\subsection{Erros de gradiente de campo}
	\pagebreak
	
\section{Oscilações em Energia}
	\subsection{Órbitas fechadas}
Nas discussões anteriores, foram analisadas trajetórias em anéis de armazenamento de elétrons com energia nominal $E_0$ --  a qual é a energia projetada para uma dada configuração ótica. No entanto, nem todos os elétrons armazenados possuem esta energia ideal. No geral, a energia $E$ de um elétron armazenado irá diferir da energia nominal, oscilando em torno desta. Estas oscilações de energia -- comumente chamadas de ''oscilações síncronas'' -- são o objeto de estudo desta seção.

Primeiramente, é necessário entender o movimento destes elétrons cuja energia difere de uma pequena quantidade $\epsilon$ da energia nominal. Mantendo a condição da \autoref{sec:3.2} de que a órbita ideal se encontra no plano horizontal, desvios de energia irão, para termos de primeira ordem, afetar apenas o movimento radial. O desvio vertical irá ser descrito apenas pelas oscilações betatron descritas na \autoref{part2}, e não serão consideradas nesta análise. Da \autoref{sec:2.6}, foi conveniente deixar que o símbolo $x$ representasse tanto $x_\beta$ quando $z_\beta$, os desvios laterais associados às oscilações betatron. A partir de agora, $x$ volta a representar o desvio horizontal total da trajetória com relação à órbita ideal.

Foi mostrado na \autoref{sec:2.5} que em um campo guia ideal o movimento radial de um elétron com um desvio de energia $\epsilon$ pode ser descrito pela soma de duas partes:
\begin{align}
	x = x_\beta + x_\epsilon
\end{align}
onde $x_\beta$ é o desvio causado pelas oscilações betatron e $x_\epsilon$ o desvio que depende apenas da energia do elétron. Acrescentando os resultados obtidos na \autoref{sec:2.11}, deve-se incluir o termo referente à distorção da órbita fechada devido às imperfeições magnéticas e escrever
\begin{align}
	x = x_\beta + x_\epsilon + x_c
\end{align}
Devido ao fato de que estes termos contribuem de forma linear -- assumindo um campo guia linear, pequenos desvios de energia e pequenas imperfeições magnéticas -- pode-se considerá-los separadamente. Agora, a análise será feita focando apenas em $x_\epsilon$.

De acordo com a equação \eqref{eq:2.28}, o desvio de energia pode ser escrito como
\begin{align}
	x_\epsilon = \eta(s)\frac{\epsilon}{E_0}
\end{align}
onde $\eta(s)$ é singularmente valorada em cada coordenada física $s$. Um elétron com energia diferente da nominal e sem oscilações betatron se move em uma nova órbita fechada onde seu desvio da órbita ideal é em todo lugar proporcional a $\epsilon/E_0$ com um fator de proporcionalidade que depende da coordenada $s$ pela função $\eta(s)$, função essa característica da configuração total do campo guia. A função $\eta(s)$ é chamada de função de dispersão, e é apenas o desvio da órbita fechada por unidade de desvio de energia.

Agora, deseja-se analisar a natureza de $\eta(s)$. $\eta(s)$ foi definida de forma que fosse a função única que satisfaz	
\begin{align}
	\begin{cases}
		\eta'' = -K_x(s)\eta + G(s), \\
        \eta(0) = \eta(L), \\
        \eta'(0) = \eta'(L).\label{eq:3.4}
    \end{cases}
\end{align}
As funções $G(s)$ e $K_x(s)$ foram definidas pelas equações \eqref{eq:2.3} e \eqref{eq:2.21}, respectivamente.

Agora, analisando o comportamento qualitativo implicado por esta definição para $\eta(s)$ de um campo guia de função separável (o qual foi definido na \autoref{sec:2.2}). Na \autoref{fig:fig29}(a),(b) estão representadas as funções $K_x$ e $G$ para um dado campo guia, e em (c) a função de dispersão $\eta(s)$.

\begin{figure}[!htb]
	\centering
	\includegraphics[width=0.6\linewidth]{./Figuras/fig29.jpeg}
	\caption{Funções do campo guia e a função de dispersão. Retirado de \cite{sands1970physics}.}
	\label{fig:fig29}
\end{figure}

Numa seção livre de campo, tanto $G$ quanto $K_x$ são nulas, então $\eta(s)$ tem um segmento com inclinação constante. Num quadrupolo puro, $G$ é zero e $K_x$ é apenas a força do quadrupolo. Num quadrupolo focalizador, $K_x$ é positivo e $\eta(s)$ segue uma oscilação senoidal em torno de zero na forma
\begin{align}
	\eta = a\ cos\left(\sqrt{K_x}s + \vartheta\right)
\end{align}
Em um quadrupolo desfocalizador, $K_x$ é negativo e $\eta(s)$ segue uma exponencial positiva na forma
\begin{align}
	\eta = a\ e^{(\sqrt{-K_x}s + \vartheta)}
\end{align}
A curva de $\eta(s)$ é "atraída" para o eixo $s$ em um quadrupolo focalizador e repelida do eixo em um quadrupolo desfocalizador.

Apesar de $K_1$ ser zero em um dipolo, $K_x$ não é. Na verdade, $K_x=G^2$ e a equação para $\eta$ fica
\begin{align}
	\eta'' = -G^2\eta + G = -G^2\left(\eta - \frac{1}{G}\right)
\end{align}

A curva de $\eta$ é um segmento senoidal o qual é "atraído" para $\eta_0 = 1/G$ com uma "força restauradora" proporcional a $G^2$ ($\eta_0$ é igual ao raio de curvatura $\rho$ da órbita ideal).

Da discussão acima, pode-se entender as características qualitativas das variações de $\eta(s)$ representadas na \autoref{fig:fig29}. Para todos os anéis de armazenamento "normais", a função de dispersão é positiva em todo o anel.

Considerando um campo guia de função separável, pode-se expandir a discussão anterior para calcular $\eta(s)$. Suponha que o cálculo inicie em $s=0$ assumindo alguns valores para $\eta(0)$ e $\eta'(0)$ e $\eta$ seja avaliada como uma sucessão de segmentos do tipo descrito anteriormente, até que seja feita uma revolução completa --  ou seja, até $s=L$. A verdadeira $\eta(s)$ será obtida se $\eta(0)$ e $\eta'(0)$ forem escolhidos de forma que $\eta(0) = \eta(L)$ e $\eta'(0) = \eta'(L)$. O cálculo pode ser computado utilizando uma técnica matricial.

A função de dispersão também pode ser obtida (para qualquer tipo de campo) utilizando os resultados obtidos na \autoref{sec:2.10} para as órbitas fechadas com distúrbio. Pode-se imaginar que a órbita fechada causada pela variação de energia é apenas uma órbita fechada com distúrbio, uma vez que tanto o desvio de energia quanto o erro de campo causam uma mudança na curvatura da trajetória. Em outras palavras, um erro de campo $\delta G$ em um segmento de órbita $\Delta s$ produz uma mudança na curvatura da trajetória de um elétron com energia $E_0$, mudança esta que é a mesma mudança na curvatura resultante de um desvio de energia $\epsilon$ de um elétron que viaja ao longo do campo nominal, já que $\delta G/G = \epsilon/E_0$. Já que $\eta(s)$ é a taxa do desvio da órbita fechada para $\epsilon/E_0$, pode-se computar $\eta(s)$ substituindo $\delta G$ na equação \eqref{eq:2.92} da \autoref{sec:2.11} por $G$. Este argumento também pode ser justificado notando que a equação \eqref{eq:3.4} para $\eta$ tem a mesma forma da equação \eqref{eq:2.85} para $x_c$ na \autoref{sec:2.10}; informalmente pode-se substituir $x_c \rightarrow \eta$ e $\delta G \rightarrow -G$. Fazendo estas substituições na equação \eqref{eq:2.94}, tem-se
\begin{align}
	\eta(s) = \frac{\sqrt{\beta(s)}}{2sen(\pi\nu)}\int\limits_{0}^{L}G(\bar{s})\sqrt{\beta(\bar{s})}cos(|\varphi(s)-\varphi(\bar{s})|-\pi\nu)d\bar{s}
\end{align}
Então, se $\beta(s)$ já é conhecido, pode-se obter $\eta(s)$ por uma integração. Note que $\eta(s)$ também terá um comportamento ressonante quando $\nu$ se aproxima de um inteiro.

Se a órbita ideal não está em um plano, esta discussão deve ser repetida para os desvios verticais. Neste caso, existirão duas funções de curvatura $G_x$ e $G_z$ assim como duas funções de focalização $K_x$ e $K_z$. Os desvios verticais também terão contribuições relativas ao desvio de energia, as quais serão proporcionais a função de dispersão $\eta_z(s)$. E a função de dispersão vertical pode ser avaliada em termos das funções de focalização e curvatura verticais. Terá apenas uma diferença qualitativa importante do caso horizontal: $\eta_z(s)$ terá tanto valores positivos quanto negativos, e sua média ao redor do anel será zero.
	\subsection{Tamanho da órbita: compactação de momento}\label{sec:3.2}
\begin{align}
	\mean{\eta}_{mag} = \frac{1}{\ell_{mag}}\int\limits_{mag}^{}\eta(s)ds\label{eq:3.13}
\end{align}
\begin{align}
	\alpha = \frac{2\pi}{L}\mean{\eta}_{mag} = \frac{\mean{\eta}_{mag}}{R}\label{eq:3.14}
\end{align}
\begin{align}
	\frac{\delta T}{T_0} = \frac{\delta \ell_\epsilon}{L} = \alpha \frac{\epsilon}{E_0}\label{eq:3.15}
\end{align}
	\pagebreak

\section{Amortecimento por Radiação}
	\pagebreak

\section{Excitação por Radiação}
	\pagebreak


\bibliography{biblio}


\end{document}








